%\documentclass[UTF8]{book} % 使用book文档类型格式排版
\usepackage{ctex}  %加载包,因为我们在用中文写文档,所以必须加载这个包,否则不支持中文

%加入了一些针对XeTeX的改进并且加入了 \XeTeX 命令来输入漂亮的XeTeX logo
\usepackage{xltxtra}
%启用一些LaTeX中的功能
\usepackage{xunicode}

\usepackage{multicol}  %加载包
\usepackage{amsmath} % 调用公式宏包
\usepackage{amssymb} % 数学符号生成命令
\usepackage{array} % 数组和表格制作
\usepackage{booktabs} % 绘制水平表格线
\usepackage{calc} %四则运算
\usepackage{caption} % 插图和表格标题格式设置
\usepackage{fancyhdr} % 页眉页脚设置
\usepackage{graphicx} % 调用插图宏包
\usepackage{multicol} % 多栏排版
\usepackage{titlesec} % 章节标题格式设置

%%%% 目录样式 %%%%
\usepackage{titletoc}
\titlecontents{chapter}[1pt]{\vspace{.5\baselineskip}\bfseries}
    {{\thecontentslabel}\quad}{}
    {\hspace{.5em}\titlerule*[10pt]{$\cdot$}\contentspage}
\titlecontents{section}[2em]{\vspace{.25\baselineskip}\bfseries}
    {\thecontentslabel\quad}{}
    {\hspace{.5em}\titlerule*[10pt]{$\cdot$}\contentspage}

\usepackage{color}
\usepackage{xcolor} % 颜色处理
%\usepackage{indentfirst} % 自动首行缩进
%\setlength{\parindent}{2.22em} % 设置首行缩进的距离
% 设置超链接颜色
\usepackage[colorlinks=true,linkcolor=black,urlcolor=black,citecolor=black]{hyperref} % 根据章节标题生成PDF书签

%%%% 版面 %%%%
\usepackage[top=0.5in,bottom=0.5in,left=1.25in,right=0.8in]{geometry}
% 设置行距
\linespread{1}
\usepackage{lscape}
\usepackage{listings} %插入代码,代码页需要加入[fragile]
\usepackage{xeCJK}

%\usepackage[slantfont,boldfont]{xeCJK} % 允许斜体和粗体

%%%% fontspec 宏包 %%%%
\usepackage{fontspec}
% 指定字体
%\setmonofont[Mapping={}]{Monaco}	%英文引号之类的正常显示,相当于设置英文字体
%\setsansfont{Monaco} %设置英文字体 Monaco, Consolas,  Fantasque Sans Mono
%\setmainfont{Monaco} %设置英文字体
% \setCJKmainfont{方正兰亭黑简体}  %中文字体设置
% \setCJKsansfont{华康少女字体} %设置中文字体
% \setCJKmonofont{华康少女字体} %设置中文字体

%%%%%%%%%% 图形支持宏包 %%%%%%%%%%
\usepackage{graphicx}                % 嵌入png图像
\usepackage{color,xcolor}            % 支持彩色文本、底色、文本框等
%\usepackage{subfigure}
%\usepackage{epsfig}                 % 支持eps图像
%\usepackage{picinpar}               % 图表和文字混排宏包
%\usepackage[verbose]{wrapfig}       % 图表和文字混排宏包
%\usepackage{eso-pic}                % 向文档的部分页加n副图形, 可实现水印效果
%\usepackage{eepic}                  % 扩展的绘图支持
%\usepackage{curves}                 % 绘制复杂曲线
%\usepackage{texdraw}                % 增强的绘图工具
%\usepackage{treedoc}                % 树形图绘制
%\usepackage{pictex}                 % 可以画任意的图形
%\usepackage{hyperref}

%\setCJKmainfont{Kai}   % 设置缺省中文字体
%\setCJKmonofont{Hei}   % 设置等宽字体
%\setmainfont{Optima}   % 英文衬线字体
%\setmonofont{Monaco}   % 英文等宽字体
%\setsansfont{Trebuchet MS} % 英文无衬线字体

\makeatletter
\providecommand*\input@path{}
\newcommand\addinputpath[1]{
\expandafter\def\expandafter\input@path
\expandafter{\input@path{#1}}}
\addinputpath{body/}
\makeatother

\definecolor{keywordcolor}{rgb}{0.8,0.1,0.5}
\lstset{language=C++, %用于设置语言为C++
    numbers=left, %设置行号位置
    numberstyle=\tiny, %设置行号大小
    keywordstyle=\color{keywordcolor} \bfseries,
    identifierstyle=,
    basicstyle=\ttfamily,
    commentstyle=\color{blue} \textit, %注释颜色
    stringstyle=\ttfamily,
    showstringspaces=false,
    frame=shadowbox, %边框
    %frame = single,
    tabsize=2, %设置tab空格数
    showspaces=false, %不显示空格
    escapeinside=``, %逃逸字符(1左面的键),用于显示中文
    %breaklines, %自动折行
    captionpos=b
}
%\begin{document}

\chapter{数学相关}

\section{Polya原理、Burnside引理}

\subsection{介绍}
群:\\
给定一个集合$G=\{a,b,c,...\}$和集合$G$上的二元运算$\star$,并满足\\
\begin{itemize}
\item 封闭性:$\forall a,b\in G,\exists c \in G,a \star b = c.$
\item 结合律:$\forall a,b,c \in G,(a \star b) \star c= a \star (b \star c).$
\item 单位元:$\exists e \in G,\forall a \in G, a \star e = e \star a = a.$
\item 逆元:$\forall a \in G,\exists b \in G, a \star b = b \star a = e,记b=a^{-1}$
\end{itemize}

则称集合$G$在运算$\star$之下是一个群,简称$G$是群。一般$a\star b$简写为$ab$。\\
置换:\\
$n$个元素$1,2,...,n$之间的置换$\begin{pmatrix} 1 & 2 & ... & n \\ a_1 & a_2  & ... & a_n\end{pmatrix}$,表示$1$被$1$到$n$中的某个数$a_1$取代,$2$被$1$到$n$中的某个数$a_2$取代,直到$n$被$1$到$n$中的某个数$a_n$取代,且$a_1,a_2,…,a_n$互不相同 \\
置换群:\\
置换群的元素是置换,运算是置换的连接。例如: \\
$\begin{pmatrix} 1& 2 & 3 & 4 \\ 3 & 1  & 2 & 4\end{pmatrix}$ $\begin{pmatrix} 1& 2 & 3 & 4 \\ 4 & 3  & 2 & 1\end{pmatrix}$ = $\begin{pmatrix} 1& 2 & 3 & 4 \\ 3 & 1  & 2 & 4\end{pmatrix}$$\begin{pmatrix} 3& 1 & 2 & 4 \\ 2 & 4  & 3 & 1\end{pmatrix}$=$\begin{pmatrix} 1& 2 & 3 & 4 \\ 2 & 4  & 3 & 1\end{pmatrix}$ \\
$Z_k $($K$不动置换类):\\
设$G$是$1…n$的置换群。若$K$是$1…n$中某个元素,$G$中使$K$保持不变的置换的全体,记以$Z_k$,叫做$G$中使$K$保持不动的置换类,简称$K$不动置换类。 \\
$E_k$(等价类):\\
设$G$是$1…n$的置换群。若$K$是$1…n$中某个元素,$K$在$G$作用下的轨迹,记作$E_k$。即$K$在$G$的作用下所能变化成的所有元素的集合。 \\
$$\mid E_k\mid· \mid Z_k\mid=\mid G\mid   \ \ \  k=1…n$$
$ D(a_j)$ 表示在置换$a_j$下不变的元素的个数 ,$s$表示置换种类数: \\
$$\sum_{i=1}^{n}\mid Z_i \mid = \sum_{i=1}^{s}D(a_i)$$

\subsection{$Burnside$引理}
 $\ L= \frac{1}{\mid G \mid}\sum_{i=1}^{n}\mid Z_i \mid = \frac{1}{\mid G \mid}\sum_{i=1}^{s}D(a_i)$,$ \ \ L$ 就是等价类数,也就是互异的组合状态的个数。\\
证明:不妨设$N={1,……,n}$中共有$L$个等价类,$N=E^1+ E^2+……+E^L$,则当$j$和$k$属于同一等价类时,有$\mid Z_j \mid = \mid Z_k \mid$。 所以\\
$$\sum_{i=1}^{n}\mid Z_i \mid = \sum_{i=1}^{L} \sum_{k \in E_i} \mid Z_k \mid = \sum_{i=1}^{L}\mid E_i \mid \cdot \mid Z_i \mid =L \cdot \mid G \mid $$

循环:\\
记$(a_1a_2...a_n)$=$\begin{pmatrix} a_1& a_2 & ... & a_{n-1} & a_n \\ a_2 & a_3   & ... & a_n & a_1 \end{pmatrix}$ 称为 $n$阶循环。每个置换都可以写成若干互不相交的循环的乘积,两个循环$(a_1a_2...a_n)$和$(b_1b_2...b_n)$互不相交是指$a_i \neq b_j,i,j=1,2,...,n$.,\\
例如:$\begin{pmatrix} 1& 2 & 3 & 4  & 5\\ 3 & 5 &  1  & 4 & 2\end{pmatrix}$ =$(1 \ 3)(2 \ 5)(4)$.这样的表示是唯一的。置换的循环节数是上述表示中循环的个数。例如$(1 \ 3)(2 \ 5)(4)$的循环节数是$3$。特别的:当$n=2$时,两阶循环$(i,j)$叫做$i$和$j$的对换。任何一个循环,都可以表达成若干换位之积。但是表达形式不尽统一,甚至连换位个数都不相同。\\
例如$(123)=(12) (13) =(13) (23) =(21)(23)=(12)(13)(31)(13)$。 尽管如此,有一个性质却是固有的,它不依换位的个数不同而异,那就是循环分解成换位的乘积时,换位个数奇偶性是不变的,或分解成奇数个换位之积或分解成偶数个换位之积。\\
若一个置换分解成奇数个换位之积,叫做奇置换;若分解成偶数个换位之积叫偶置换。单位置换为偶置换。\\

\subsection{Polya原理}

设$G$是$p$个对象的一个置换群,用$m$种颜色涂染$p$个对象,$G=\{g_1,g_2,...,g_s\}$,$g_i$是不同的置换选择,令$g_i$的循环节数为$c(g_i)\ (i \ \in [1,s])$,则不同的染色方案数为:$L \ = \frac{1}{\mid  G \mid}(m^{c(g_1)}\ + m^{c(g_2)}\ +... + m^{c(g_s)})$ \\

\subsection{$n*n$的方阵,每个小格可涂$m$种颜色,求在旋转操作下本质不同的解的总数}

我们可以在方阵中分出互不重叠的长为$[\frac{n+1}{2}]$,宽为$[\frac{n}{2}]$的四个矩阵。当$n$为偶数时,恰好分完;当$n$为奇数时,剩下中心的一个格子,它在所有的旋转下都不动,所以它涂任何颜色都对其它格子没有影响。令$m$种颜色为$0\sim m-1$,我们把矩阵中的每格的颜色所代表的数字顺次(左上角从左到右,从上到下;右上角从上到下,从右到左;……)排成$m$进制数,然后就可以表示为一个十进制数,其取值范围为 $0 \sim m^{\frac{n^2}{4}}-1$。(因为$\frac{n}{2}*\frac{n+1}{2}=\frac{n^2}{4}$) 这样,我们就把一个方阵简化为$4$个整数。我们只要找到每一个等价类中左上角的数最大的那个方案(如果左上角相同,就顺时针方向顺次比较) 这样,在枚举的时候其它三个数一定不大于左上角的数,效率应该是最高的。  进一步考虑,当左上角数为$i$时,($i \in [0, R-1]$). 令$R=m^{\frac{n^2}{4}}$ 可分为下列的4类:
\begin{enumerate}
\item  其它三个整数均小于$i$,共$i^3$个。
\item 右上角为$i$,其它两个整数均小于$i$,共$i^2$个。
\item  右上角、右下角为$i$,左下角不大于$i$,共$i+1$个。
\item  右下角为$i$,其它两个整数均小于$i$,且右上角的数不小于左下角的,共$\frac{i\ *\ (i+1)}{2}$个。
\end{enumerate}

因此:
\[
\begin{aligned}
L&=\sum_{i=0}^{R-1}(i^3+i^2+i+1+\frac{1}{2}i(i+1))=\sum_{i=0}^{R-1}(i^3+\frac{3}{2}i^2+\frac{3}{2}i+1) \\
&=\sum_{i=1}^{R}((i-1)^3+\frac{3}{2}(i-1)^2+\frac{3}{2}(i-1)+1)=\sum_{i=1}^{R}(i^3-\frac{3}{2}i^2+\frac{3}{2}i) \\
&=\frac{1}{4}R^2(R+1)^2-\frac{3}{2}*\frac{1}{6}R(R+1)(2R+1)+\frac{3}{2}*\frac{1}{2}R(R+1) \\
&=\frac{1}{4}(R^4+R^2+2R) \\
\end{aligned}
\]

当$n$为奇数时,还要乘以一个$m$. \\

\underline {采用$Polya$原理解决} \\
确定置换群: \\
只有$4$个置换:$转0^\circ ,转90^\circ,转180^\circ,转270^\circ$,所以置换群$G=\{转0^\circ ,转90^\circ,转180^\circ,转270^\circ\}$ \\
计算循环节个数: \\
首先,给每个格子顺次编号$(1\sim n^2)$,再开一个二维数组记录置换后的状态。最后通过搜索计算每个置换下的循环节个数,效率为一次方级。\\
代入公式 : \\
利用$Pólya$定理得到最后结果。\\
$$L \ = \frac{1}{\mid  G \mid}(m^{c(g_1)}\ + m^{c(g_2)}\ +... + m^{c(g_s)})$$
当然也可以直接思考循环节: \\
\begin{itemize}
\item 当$n$为偶数,在转$0^\circ$时,循环节为$n^2$个,转$180^\circ$时,循环节为$\frac{n^2}{2}$个,转$90^\circ$和$270^\circ$时,循环节为$\frac{n^2}{4}$ 个
\item 当$n$为奇数,在转$0^\circ$时,循环节为$n^2$个,转$180^\circ$时,循环节为$\frac{n^2 - 1}{2} + 1$个,转$90^\circ$和$270^\circ$时,循环节为$\frac{n^2 - 1}{4}+1$个
\end{itemize}
综合考虑可得:$L \ = \frac{1}{4}(m^{n^2} + m^{\frac{n^2+3}{4}}\ +m^{\frac{n^2+1}{2}} + m^{\frac{n^2+3}{4}})$。可以发现这个式子,其实是和数学推导的式子完全吻合的。
\begin{lstlisting}
typedef long long ll;
const int MAX_N = 10;
const ll mod = 1000000007;

int a[MAX_N][MAX_N], b[MAX_N][MAX_N];
int m, n; // m: 颜色数, n: 方阵大小
ll ans;

void Rotate() //逆时针旋转 90 度
{
	for(int i = 1; i <= n; ++i) {
		for(int j = 1; j <= n; ++j) {
			a[n - j + 1][i] = b[i][j];
		}
	}
	for(int i = 1; i <= n; ++i) {
		for(int j = 1; j <= n; ++j) {
			b[i][j] = a[i][j];
		}
	}
}

void CircleSection() //计算当前状态循环节数
{
	int num = 0; //记录循环节个数
	for(int i = 1; i <= n; ++i) {
		for(int j = 1; j <= n; ++j) {
			if(a[i][j] > 0) { //搜索尚未被访问过的格子
				num++;
				int nexti, nextj, p;
				p = a[i][j];
				a[i][j] = 0;
				nexti = (p - 1) / n + 1;
				nextj = (p - 1) % n + 1; //得到这个循环的下一个格子
				while(a[nexti][nextj] > 0) {
					p = a[nexti][nextj];
					a[nexti][nextj] = 0; //已访问格子置零
					nexti = (p - 1) / n + 1;
					nextj = (p - 1) % n + 1;
				}
 			}
		}
	}
	ll res = 1;
	for(int i = 1; i <= num; ++i ) {
		res = res * m % mod;
	}
	ans = (ans + res) % mod;
}

void init()
{
	for(int i = 1; i <= n; ++i) {
		for(int j = 1; j <= n; ++j) {
			b[i][j] = a[i][j] = (i - 1) * n + j;
		}
	}
}

int main()
{
	ll tt = quick_pow(4, mod - 2); //求除以 4 模 mod 的逆元
	while(~scanf("%d%d", &n, &m)){
		ans = 0;
		init(); //对方阵状态进行初始化
		CircleSection(); //旋转 0 度状态下的循环节个数
		Rotate();
		CircleSection();//逆时针 90 度
		Rotate();
		CircleSection(); //逆时针 180 度
		Rotate();
		CircleSection();//逆时针 270 度
		Rotate();
		ans = ans * tt % mod;
		printf("ans = %lld\n", ans); //根据 Polya 原理得到的结果

		ll R = quick_pow(m, n * n / 4);
		ll res = R * R % mod * R % mod * R % mod + R * R % mod + 2 * R % mod ;
		if(n & 1) res = res * m % mod;
		res = res * tt % mod;
		printf("res = %lld\n", res); //根据数学推导得到的结果
	}
	return 0;
}
\end{lstlisting}

\subsection{各有$a,b,c(a,b,c \geq  0, a+b+c \leq 40)$颗三种颜色,问这些珠子能串成的项链有多少种?考虑翻转和旋转。}
\underline {[UVA 11255 Necklace]} \\
令$\sum_{i=1}^{3}color[i]=n$,即珠子总数。 \\

考虑旋转置换 \\

我们考虑旋转$i$颗珠子的间距,则$0,i,2i...$构成一个循环,这个循环有$\frac{n}{gcd(n,i)}$.根据对成性,所有循环的长度均相同,因此一共有$gcd(i,n)$个循环。循环与循环之间是等价类,所以在一个循环内的某颗珠子的颜色确定了,那么在其余循环内的同样地位位置的珠子颜色也就确定了。我们得到的答案是每个循环的方案数并且在旋转置换下,当循环节确定了,每个循环的方案数都是一样的 。我们假设循环节的长度(也就是循环内元素的个数)为$x\in [0,n)$,第$i$种颜色的珠子的个数为$color[i]$个,如果$color[i] \% x !=0$,那么第$i$种颜色的珠子在每个循环(等价类)内就不能均分,所以这种循环节就应该摒弃。当$color[i] \% x ==0$时,令$b[i] = \frac{color[i]}{x}$,则$b[i]$就是每个循环(等价类)分得的这种颜色的珠子的个数。显然每个循环(等价类)内共有$\sum_{i=1}^{3}b[i]=\frac{1}{x}\sum_{i=1}^{3}color[i]$颗珠子,记为$sum$. 我们考虑单个循环(等价类):有$sum$颗珠子,各个颜色分别有$b[i]$颗,那么根据排列组合在循环节为$x$时,可得$$Ans[x]=C_{sum}^{b[0]}*C_{sum-b[0]}^{b[1]}*C_{sum-b[0]-b[1]}^{b[2]}$$

考虑翻转置换 \\

当$n$为奇数时,则对称轴上必有一点,对称轴有$n$条,每条对称轴形成$\frac{n-1}{2}$个长度为$2$的循环和$1$个长度为$1$的循环。\\

当$n$为偶数时,有两种对称轴。穿过珠子的对称轴有$\frac{n}{2}$条,各形成$\frac{n}{2}-1$个长度为2的循环和两个长度为$1$的循环。不穿过珠子的对称轴也有$\frac{n}{2}$条,各形成$\frac{n}{2}$个长度$2$的循环。\\

可以发现不论$n$为奇偶,对称轴的总个数都为$n$,同时旋转置换的个数也为$n$,所以根据$Burnside$引理最后需要除以$2n$.

\begin{lstlisting}
const int MAX_N = 42;

ll C[MAX_N][MAX_N];
int color[5], b[5], n;

void init() //组合数打表
{
	C[0][0] = 1;
	for(int i = 1; i < MAX_N; ++i) {
		C[i][0] = C[i][i] = 1;
		for(int j = 1; j < i; ++j) {
			C[i][j] = C[i - 1][j] + C[i - 1][j - 1];
		}
	}
}

int gcd(int x, int y)
{
	return y == 0 ? x : gcd(y, x % y);
}

ll CircleSection(int x)
{
	int sum = 0;
	for(int i = 1; i <= 3; ++i) {
		if(b[i] % x) return 0;
		b[i] /= x;
		sum += b[i];
	}
	ll res = 1;
	for(int i = 1; i <= 3; ++i) {
		res *= C[sum][b[i]];
		sum -= b[i];
	}
	return res;
}

ll Rotate() //旋转置换
{
	ll res = 0;
	for(int i = 0; i < n; ++i) {
		int d = gcd(i, n);
		memcpy(b, color, sizeof(color));
		res += CircleSection(n / d); // n/d 是循环元素个数
	}
	return res;
}

ll Flip() //翻转置换
{
	ll res = 0;
	if(n & 1) {
		for(int i = 1; i <= 3; ++i) {
			memcpy(b, color, sizeof(color));
			b[i]--;
			if(b[i] < 0) continue;
			res += CircleSection(2) * n; // n 条对称轴
		}
	}else {
		//穿过珠子
		for(int i = 1; i <= 3; ++i) {
			for(int j = 1; j <= 3; ++j) {
				memcpy(b, color, sizeof(color));
				b[i]--, b[j]--;
				if(b[i] < 0 || b[j] < 0) continue;
				res += CircleSection(2) * (n / 2);
			}
		}
		//不穿过珠子
		memcpy(b, color, sizeof(color));
		res += CircleSection(2) * (n / 2);
	}
	return res;
}

int main()
{
	init();
	int T;
	scanf("%d", &T);
	while(T--) {
		n = 0;
		for(int i = 1; i <= 3; ++i) {
			scanf("%d", color + i);
			n += color[i];
		}
		ll ans = 0;
		ans += Rotate();
		ans += Flip();
		printf("%lld\n", ans / (2 * n));
	}
	return 0;
}
\end{lstlisting}

\subsection{给出12根等长的火柴棒,每根火柴棒的颜色属于$1-6$中的一种,问能拼成多少种不同的正方体?(考虑旋转)}

\underline {[UVA 10601 Cubes]}首先正方体的旋转置换有$24$种。下面将每个循环内元素的个数称为循环的长度。\\

\underline{注意是棱边的置换循环,而不是面的置换循环}.\\
\begin{enumerate}
\item 静止。只有一种置换。有$12$个循环,每个循环的长度为$1$。
\item 以相对面的中心为轴旋转。可以旋转的角度是$90^{\circ},180^{\circ},270^{\circ}$。 选择轴有3种,每种轴下有3种旋转,所以有3*3=9 种置换。
\item 旋转$90^{\circ},270^{\circ}$ ,都是三个循环,每个循环的长度为$4$.
\item 旋转$180^{\circ}$,有6个循环,每个循环的长度为2
\item 以对边中点为轴,只可以旋转$180^{\circ}$,选择轴有6种,所以有6*1=6种置换。在这种旋转下有5个长度为2的循环和2个长度为1的循环
\item 以对顶点为为轴,可以旋转$120^{\circ},240^{\circ}$。 选择轴有4种,所以有4*2=8种置换。
\item 旋转$120^{\circ},240^{\circ}$都是有4个长度为3的循环。
\end{enumerate}
找到了在每种置换下的循环个数和每个循环的长度就可以参考前面【例2】处理旋转置换的方法解决。 \\
需要特别关注的是,\underline {以对边中点为轴,只可以旋转$180^{\circ}$,有5个长度为2的循环和2个长度为1的循环。} \\
我们可以先枚举两个长度为1的循环选择的颜色,然后对于5个长度为2的循环就和上面的一样了。最后别忘了根据$Burnside$引理需要除以总的置换数$24$。

\begin{lstlisting}
ll work(int k)
{ //每 k 条边必须相同,分成 12/k 组(以对边中点为轴旋转 180° 是分成 5 组)
	memcpy (b, a, sizeof(a));
	int sum = 0;
	for (int i = 1; i <= 6; ++i) {
		if(b[i] % k) return 0;
		b[i] /= k;
		sum += b[i];
	}
	ll res = 1;
	for (int i = 1; i <= 6; ++i) {
		res *= C[sum][b[i]];
		sum -= b[i];
	}
	return res;
}

ll solve()
{
	ll res = 0;
	//静止
	res += work(1);
	//以相对面中心为轴
	res += (ll)3 * 2 * work(4); //旋转 90° 和270°
	res += (ll)3 * work(2); //旋转180°
	// 以对顶点为轴,可以旋转 120° 或 240°
	res += (ll)4 * 2 * work(3);
	// 以对边种点为轴,只能旋转 180°
	for(int i = 1; i <= 6; ++i) {
		for(int j = 1; j <= 6; ++j) {
			if(a[i] == 0 || a[j] == 0) continue;
			a[i]--; a[j]--;  // 将 a[i] 和 a[j] 设为选择的两条对边的颜色
			res += (ll)6 * work(2);
        //剩下的是 5 个循环长度为 2 的循环, 6 代表对边选择情况
			a[i]++; a[j]++;
		}
	}
	return res / 24;
}
\end{lstlisting}

\subsection {[POJ 2888 Magic Bracelet]}

有一串$n$个珠子的项链,用$m$种颜色来染,有$k$个限制条件:$a[i]$和$b[i]$不能相邻。问本质不同的项链有多少种?(考虑旋转,答案对$9973$取模,且$gcd(n,9973)=1$)。数据范围:$n\leq 10^9,1\leq m \leq 10,0 \leq k \leq \frac{m*(m-1)}{2}$ \\

假设有$k$个循环,用$link[i][j]$表示第$i$种颜色能否和第$j$种颜色相邻:当$link[i][j]=1$,表示$i$和$j$不能相邻,否则可以相邻。无向边:$link[i][j]=link[j][i]$ 。 \\
用$dp[k][i][j]$表示经过$k$个循环从第$i$种颜色转移到第$j$种颜色的方案数: \\
$$dp[k][i][j]=\sum_{p\leq m}(1-link[p][j])*dp[k-1][i][p]$$
初始化:$dp[1][i][j] = 1-link[i][j]$ \\
初始矩阵:$A[i][j]=1-link[i][j]$来表示颜色间的连通性 \\
由上面的递推式,$A^k[i][j]$代表的是:从第$i$种颜色经过$k$个循环后变为第$j$种颜色的方法数。考虑循环,需要知道从$i$ 出发回到$i$ 的方案数即:$A^{k}[i][i]$。$solve(k)$表示循环个数为$k$时的方案数: \\
$$solve(k)=\sum_{i\leq m}A^{k}[i][i]$$

离散数学里有:如果用0,1矩阵$A$来表示无向图的连通情况的话,$A^k$代表的就是一个点经过$k$条路后能到达的地方的方法数。\\

假设对于每个循环的步长为$i$,也就是$0,i,2i,...$构成一个循环。这个循环的周期为$\frac{i*n}{gcd(i,n)}$,所以这个循环有$\frac{n}{gcd(i,n)}$ 个元素,共有$gcd(i,n)$个循环。所以枚举的循环个数一定是$n$的因子,即:$k \ |\ n$。\\
满足循环个数为$k$的置换的旋转步长$i$满足$gcd(i,n)=k$,此种置换的个数也就是$gcd(\frac{i}{k},\frac{n}{k})=1$的$i$的个数,即:$\phi(\frac{n}{k})$.\\

综上对于每个满足$n \% k = 0$的$k$可以得到的方案数是\\
$$solve(k)*\phi (\frac{n}{k})$$
枚举每个$k$然后求和最后还要除以$n$(因为总的置换数为$n$),又因为有模数且模数为素数,那么就相当于乘以$n^{mod-2}$(费马小定理)。
$$Ans = \{ \sum_{i \mid n}solve(i)*\phi(\frac{n}{i})\} * n^{p - 2} \ \% p$$
\begin{lstlisting}
const ll mod = 9973;

int link[15][15];

struct Matrix{
	int row, col;
	ll data[15][15];

	Matrix operator * (const Matrix& rhs) const { // 矩阵相乘条件:col = rhs.row
		Matrix res;
		res.row = row, res.col = rhs.col;
		for(int i = 1; i <= res.row; ++i) {
			for(int j = 1; j <= res.col; ++j) {
				res.data[i][j] = 0;
				for(int k = 1; k <= row; ++k) {
					res.data[i][j] += data[i][k] * rhs.data[k][j];
				}
				res.data[i][j] %= mod;
			}
		}
		return res;
	}

	Matrix operator ^ (const int m) const {  // 矩阵快速幂
		Matrix res, tmp;
		res.row = row, res.col = col; //row == col
		memset(res.data, 0, sizeof(res.data));

		tmp.row = row, tmp.col = col;
		memcpy(tmp.data, data, sizeof(data));
		for(int i = 1; i <= res.row; ++i) { res.data[i][i] = 1; }
		int mm = m;
		while(mm) {
			if(mm & 1) res = res * tmp;
			tmp = tmp * tmp;
			mm >>= 1;
		}
		return res;
	}
};

inline ll solve(int len, int m)
{
	Matrix tmp;
	tmp.row = tmp.col = m;
	for(int i = 1; i <= m; ++i) {
		for (int j = 1; j <= m; ++j) {
			tmp.data[i][j] = 1 - link[i][j];
		}
	}
	tmp = tmp ^ len;

	ll ans = 0;
	for (int i = 1; i <= m; ++i) {
		ans = (ans + tmp.data[i][i]) % mod;
	}
	return ans;
}

int main()
{
	int T, n, m, t;
	scanf("%d", &T);
	while (T--) {
		memset(link, 0, sizeof(link));
		scanf("%d%d%d", &n, &m, &t);
		for(int i = 0; i < t; ++i) {
			int former, later;
			scanf("%d%d", &former, &later);
			link[former][later] = link[later][former] = 1;
		}

		ll ans = 0;
		for(int i = 1; i * i <= n; ++i) {
			if(n % i) continue;
			ans = (ans + phi(n / i) * solve(i, m)) % mod;
			if(n / i == i) continue;
			ans = (ans + phi(i) * solve(n / i, m)) % mod;
		}
		printf("%lld\n", ans * quick_pow(n % mod, mod - 2, mod) % mod);
	}
	return 0;
}
\end{lstlisting}

如果仅仅限制相邻的珠子颜色不能一样,总共有$m$种颜色。对于步长为$k$的循环,定义$dp[i][0]$表示经历$i$个循环珠子颜色和循环起始珠子颜色一致的方案数,定义$dp[i][1]$表示颜色不一样(初始化$dp[1][0]=1,dp[1][1]=0$),那么可以得到转移方程:\\
\begin{center}
$dp[i][0]=dp[i-1][1]$\\
$dp[i][1]=dp[i-1][1]*(m-2)+dp[i][0]*(m-1)$
\end{center}

如果$m$很大的话,就用矩阵快速幂进行加速。定义矩阵:$A=\begin{pmatrix} 0 & 1 \\ m - 1 & m - 2\end{pmatrix},$$B=\begin{pmatrix} 1 \\ 0\end{pmatrix}$,$C=A^{k}*B$.那么步长为$k$的循环总的方案数(考虑起始珠子有$m$种颜色可选)是:
$$\phi(\frac{n}{k})*m*C[1][1]$$
条件是$n\%k=0$, 最后还要乘上$n$模$mod$的逆元。

\clearpage
\section{容斥原理}
\subsection{求区间$[A,B]$中和$n$互素的数个数?$1\leq A \leq B \leq 10^{15},1\leq n \leq 10^9$}

\underline {[HDU 4135]} \\
我们将$n$进行素因子分解为$p_1,p_2, p_3, \cdots, p_k$,先求解$[1,r]$中和$n$不互素的数字个数。我们只需要考虑最大公约数是素因子的倍数的情况。选择素因子$num$个相乘,得到$mul$,计算$[1,r]$中和$n$不互素的数字个数是根据$num$的奇偶性:奇加偶减,$\frac{r}{mul}$。 \\
有位运算和$dfs$两种写法。
\begin{lstlisting}
vector<ll> fac;

void GetFactor(int n)
{
	fac.clear();
	for(ll i = 2; i * i <= n; ++i) {
		if(n % i == 0) {
			fac.push_back(i);
			while(n % i == 0) n /= i;
		}
	}
	if(n > 1) fac.push_back(n);
}
/*
ll solve(ll A, int n)
{
	ll ans = 0;
	int total = fac.size();
	for(ll i = 1; i < (1 << total); ++i) { // 用二进制位表示该位上对应编号的素因子是否选择
		int bits = 0;
		ll res = 1;
		for(int j = 0; j < total; ++j) {
			if(i & (1 << j)) { // 如果选择了第 j 个素因子
				bits++;
				res *= fac[j];
			}
		}
		if(bits & 1) ans += A / res; // 选择的素因子个数为奇数个
		else ans -= A / res;
	}
	return A - ans;
}
*/
ll ans;

void dfs(int cur, int num, ll mul, ll A)
{
	if(cur == fac.size()) {
		if(num & 1) ans -= A / mul;
		// A / mul 是最大公约数是 mul 的数字个数, ans 求的是互素数个数
		else ans += A / mul;
		return;
	}
	dfs(cur + 1, num, mul, A);
	dfs(cur + 1, num + 1, mul * fac[cur], A);
}

int main()
{
	int T, n, cases = 0;
	ll A, B;
	scanf("%d", &T);
	while(T--) {
		scanf("%lld%lld%d", &A, &B, &n);
		GetFactor(n);
		ans = 0;
		dfs(0, 0, 1, B);
		ll res = ans;
		ans = 0;
		dfs(0, 0, 1, A - 1);
		printf("Case #%d: %lld\n", ++cases, res - ans);
		//printf("Case #%d: %lld\n", ++cases, solve(B, n) - solve(A - 1, n));
	}
	return 0;
}
\end{lstlisting}

\subsection{给一个$n$,在$p\in [1, n]$范围满足$m^k=p(m \geq 1, k > 1$的数字$p$的个数。$1 \leq n \leq 10^{18}$}
\underline {[HDU 2204]}
我们可以枚举幂次$k$,考虑到$2^{60}>10^{18}$,最多只需要枚举到$60$幂次。 \\
同时对于一个数$p$的幂次$k$是个合数,那么$k$一定可以表示成$k=r*k'$,其中$k'$是素数的形式,那么: \\
$$p=m^k=m^{r*k'}=(m^r)^{k'}$$
所以我们只需要枚举素幂次$k$即可。\\
同时如果$p^k \leq n$,那么对于任意的$p' < p$,也一定满足${p'}^{k} \leq n$。所以对于每个$k$我们令$p^k=n$,即$p=n^{\frac{1}{n}}$,求出最大的$p$,同时也就是满足$p^k\leq n$的所有$p$的个数。
但是这样子会有重复。例如:$k=2$时,${2^2}^3$和$k=3$时,${2^3}^{2}$就重复计数了。这时候需要用容斥原理:加上奇数个素幂次相乘的个数,减去偶数个素幂次相乘的个数。 \\
又因为$2*3*5 < 60,2*3*5*7 > 60$,那么最多只要考虑三个素幂次相乘情况。 \\
时间复杂度:$O(3*2^{17})$(60以内共17个素数)
\begin{lstlisting}
const ll prime[20] = {2, 3, 5, 7, 11, 13, 17, 19, 23, 29, 31, 37, 41, 43, 47, 53, 59};
const int len = 17;
const double eps = 1e-8;

ll ans;
double n;

void dfs(int cur, int num, int total, ll k)
{
	if(k > 60) return; // 素因子连乘最多不能超过 60 次幂,因为2 ^ 60 > 10 ^ 18
	if(num == total) {
		ll p = (ll)(pow(n, 1.0 / (0.0 + k)) + eps) - 1; // 先把 1 去掉, eps精度误差
		ans += p;
		return ;
	}
	if(cur == len) return ;
	dfs(cur + 1, num, total, k); //第 i 个素数不选
	dfs(cur + 1, num + 1, total, k * prime[cur]); //第 i 个素数选择
}

int main()
{
	while(~scanf("%lf", &n)) {
		ll res = 0;
		for(int i = 1; i <= 3; ++i) {
			ans = 0;
			dfs(0, 0, i, 1);
//从下标 0 开始,当前选择素数个数为 0 ,需要选择素数个数 i 个,选择素数乘积为1
			if(i & 1) res += ans;
			else res -= ans;
		}
		res += 1; // 1在 dfs 时都没有统计
		printf("%lld\n", res);
	}
	return 0;
}
\end{lstlisting}

\clearpage
\section{博弈论}

\subsection{威佐夫博弈$(Wythoff\quad  Game)$}
有两堆石子各有$a,b$个,两个人轮流从某一堆或同时从两堆中取同样多的物品,规定每次至少取一个,多者不限,最后取光者得胜。判断先手是胜者还是负者。 \\

我们用 $(a_k,b_k),a_k\leq b_k,k=0,1,2,…,n$ 表示两堆物品的数量并称其为局势,如果甲面对$(0,0)$,那么甲已经输了,这种局势我们称为奇异局势。前几个奇异局势是:$(0,0)$,$(1,2)$,$(3,5)$$(4,7)$,$(6,10)$,$(8,13)$,$(9,15)$,$(11,18)$,$(12,20)$。 \\
可以看出,$a_0=b_0=0$,$a_k$ 是未在前面出现过的最小自然数,而$b_k=a_k+k$,奇异局势有如下三条性质:
\begin{enumerate}
\item 任何自然数都包含在一个且仅有一个奇异局势中。由于$a_k$ 是未在前面出现过的最小自然数,所以有$a_k>a_{k−1}$,而 $b_k=a_k+k>a_{k−1}+k−1=b_{k−1}>a_{k−1}$。 所以性质1成立。
\item 任意操作都可将奇异局势变为非奇异局势。事实上,若只改变奇异局势$(a_k,b_k)$的某一个分量,那么另一个分量不可能在其他奇异局势中,所以必然是非奇异局势。 如果使$(a_k,b_k)$的两个分量同时减少,则由于其差不变,且不可能是其他奇异局势的差,因此也是非奇异局势。
\item 采用适当的方法,可以将非奇异局势变为奇异局势。
\end{enumerate}
从如上性质可知,两个人如果都采用正确操作,那么\underline{面对非奇异局势,先拿者必胜;反之,则后拿者取胜}。 \\
任给一个局势 $(a,b)$,判断它是不是奇异局势,有如下公式: \\
$$a_k=\lfloor k\times \frac{1+\sqrt{5}}{2}\rfloor,b_k=a_k+k\quad (k=0,1,2,…,n)$$
由于$\frac{1+\sqrt{5}}{2}=\frac{2}{\sqrt{5} - 1}$,可以先求出 $j=\lfloor a\times \frac{\sqrt{5}-1}{2}\rfloor$,则$a_j=\lfloor j \times \frac{\sqrt{5}+1}{2}\rfloor$,
\begin{itemize}
\item 若$a=\lfloor j \times \frac{\sqrt{5}+1}{2}\rfloor=a_j$那么 判断$b$是否等于$b_j=a_j+j=a+j$
\item 若$a$不等于$a_j$,那么判断是否满足$a=a_{j+1}=\lfloor (j+1) \times \frac{\sqrt{5}+1}{2}\rfloor$,$b=b_{j+1}=a_{j+1}+j+1$
\item 若都不是,那么就不是奇异局势。
\end{itemize}

\begin{lstlisting}
double p = (sqrt(5.0) + 1.0) / 2.0;
if(a > b) swap(a, b);
int k = (int)(a / p);
if((a == (int)(k * p) && b == a + k) ||
	(a == (int)((k + 1) * p) && b == a + k + 1)) {
	printf("0\n"); // 先手输
} else {
	printf("1\n"); // 先手胜
}
\end{lstlisting}

\clearpage
\section{约瑟夫环问题}
$n$个人顺时针站成一圈,编号$0\sim n-1$,初始位置为0,每次让第$m$个人出列,问最后剩下的人的编号。 \\

定义当有$n$个人围成一圈时最终剩下人的编号为$dp[n]$,显然有$dp[1]=0$。考虑$dp[n]$向$dp[n-1]$的转化。 \\
当有$n$个人时,编号顺序为:
$$
0\quad 1\quad 2\quad \dots\quad m-2\quad m-1\quad m \dots \quad n-2 \quad n-1
$$
起始位置是0,第$m$个人(编号为$m-1$)出列后,变为:
$$
0\quad 1\quad 2\quad \dots\quad m-2\qquad \qquad m\quad \dots \quad n-2\quad n-1
$$
现在只剩下了$n-1$个人。对比$n-1$个人的时候的初始情况,把$n-1$个人的0编号和$n$个人剔除$m-1$编号后的$m$编号对齐,把$n-1$个人的1编号和$n$个人剔除$m-1$编号后的$m+1$编号对齐。。。依此类推可以由当$n-1$个人围成一圈最终剩下的人编号$dp[n-1]$得到递推式:
$$dp[n]=(dp[n-1]+m)\% n$$

\subsection{第一次第$m$个人出列,以后每次第$k$个人出列}
\underline {PKU 1781}\\

注意到状态定义:$dp[n]$表示$n$个人围成一圈最终剩下的人的编号(编号从0开始)。因为第一次第$m$个人出列的时候是$n$个人围成一圈,所以有:$dp[n]=(dp[n-1]+m)\%n$。而当$i < n$个人围成一圈时,每次都是第$k$个人出列,所以:$dp[i]=(dp[i-1]+k)\%i\quad (i < n)$。编号从0开始。

\begin{lstlisting}
dp[1] = 0;
for (int i = 2; i < n; ++i) {
	dp[i] = (dp[i - 1] + K) % i;
}
dp[n] = (dp[n - 1] + m) % n;
\end{lstlisting}

\subsection{每次都是第2个人出列,求$n$个人最终剩下的编号(编号从1开始)}
\underline {PKU 1781} \\

$n\leq 10^{18}$。递推式:

\begin{lstlisting}
dp[1] = 1
dp[i] = (dp[i - 1] + 2) % i  (i > 1)
if (dp[i] == 0) dp[i] = i;
\end{lstlisting}

打表找规律可以发现:
\[
\begin{aligned}
dp[1] &= 1 \\
dp[2*i] &= 2 * dp[i] - 1 \\
dp[2*i+1] &= 2 * dp[i] + 1
\end{aligned}
\]
递归解决即可。

\subsection{第$i$轮从上一轮出局的人的下一个人开始从$1$报数,报到$i$就停止且报到$i$的这个人出局}

\underline {HDU 5643}

$n\leq 5000$围成一圈,编号依次为$1\sim 5000$,第一轮第一个人从1开始报数,报到1就停止且报到1的这个人出局。第二轮从上一轮出局的人的下一个人开始从1报数,报到2就停止且报到2的这个人出局。第三轮从上一轮出局的人的下一个人开始从1报数,报到3就停止且报到3的这个人出局。依次类推,求最终剩下的人的编号。\\

先把下标都从0开始,最后再统一加上1。用$dp[i][j]$表示当一共$i$个人围成一圈并且第一个人开始报数$j$得时候最终剩下的人的编号。初始化:$dp[1][]=0$。状态转移:
\[
\begin{aligned}
dp[i][j] &=(dp[i-1][j+1]+j)\% i \\
ans[i] &=dp[i][1]+1
\end{aligned}
\]
$O(n^2)$预处理即可。

\subsection{获得每一轮出列的人的编号}

\underline {PKU 2886 和 HDU 5860}\\

每个人有一个下一次出列的指向性:$data[i]$表示当第$i$个人出列时下一个出列的人是从他左起$(data[i] > 0)$或者右起$(data[i]<0)$ 的第$|data[i]|$个人,第一次第$k$个人出列。 \\

借助线段树解决。叶子结点0/1表示当前编号的人是否已经出列。区间表示当前区间剩余未出列的人数。当一个人出列时,求出下一个出列的人是剩下人中的第几个。区间查询,单点更新。\\
时间复杂度:$O(n\log n)$。\\

[HDU 5860]:现在有$n\leq 3*10^{6}$个人,编号为$1\sim n$,依次排列成一列,给定一个$k$,每一轮依次为当前排列的第1、第$k+1$、第$2k+1$…直到列尾的人出局,现在有$Q\leq 10^5$个询问,每个询问为$ack[i]$,问第$ask[i]$个出局的人的编号。\\
例如:$n=7,k=3$时出局的人的编号依次是:
$$1\quad 4\quad 7\quad 2\quad 5\quad 3\quad 6$$

\begin{lstlisting}
const int MAX_N = 3000010;

int T, n, K, Q, Max;
int ans[MAX_N], stree[MAX_N << 2], ask[1000010];

void build(int left, int right, int cur)
{
	if (left == right) {
		stree[cur] = 1;
		return;
	}
	int mid = (left + right) >> 1;
	build(left, mid, lson(cur));
	build(mid + 1, right, rson(cur));
	stree[cur] = stree[lson(cur)] + stree[rson(cur)];
}

int query(int pos, int left, int right, int cur)
{
	if (left == right) return left;
	int mid = (left + right) >> 1;
	if (pos <= stree[lson(cur)]) return query(pos, left, mid, lson(cur));
	else return query(pos - stree[lson(cur)], mid + 1, right, rson(cur));
}

void update(int pos, int left, int right, int cur)
{
	if (left == right) {
		stree[cur] = 0;
		return;
	}
	int mid = (left + right) >> 1;
	if (stree[lson(cur)] >= pos) update(pos, left, mid, lson(cur));
	else update(pos - stree[lson(cur)], mid + 1, right, rson(cur));
	stree[cur] = stree[lson(cur)] + stree[rson(cur)];
}

void solve()
{
	build(1, n, 1);
	int total = 0, left = n;
	while (left) {
		int i = 1, cnt = 0;
		while (i <= left) {
			ans[++total] = query(i, 1, n, 1);
			cnt++;
			i += K;
			if (total >= Max) return;
		}
		i -= K;
		while (i >= 1) {
			update(i, 1, n, 1);
			i -= K;
		}
		left -= cnt;
	}
}

int main()
{
	scanf("%d", &T);
	while (T--) {
		scanf("%d%d%d", &n, &K, &Q);
		Max = 0;
		for (int i = 0; i < Q; ++i) {
			scanf("%d", &ask[i]);
			Max = max(Max, ask[i]);
		}
		solve();
		for (int i = 0; i < Q; ++i) {
			printf("%d\n", ans[ask[i]]);
		}
	}
	return 0;
}
\end{lstlisting}

\clearpage
\section{康托展开}
\underline {[UVALive 6665 Dragonas Cruller]} \\
以九宫格的形式给出0--8八个数字,然后通过移动0数字,使这个九宫格变成给定的状态,上下移动和左右移动的权值不一样,求最小移动路径值。\\

用康拓排序来去重。因为上下移动和左右移动的权值不一样,所以必须使用优先队列,这样才能保证解的优先性。
\begin{lstlisting}
const int MAX_N = 400000;

int ch, cv, ans, st, ed, pos, npos;
int ast[15], aed[15], temp[15], fac[15] = {1, 1};
int vis[MAX_N];

struct Node{
    int cost, pos, cantor;
    // 当前状态下的路径值, 0 位置和康拓值
    bool operator < (const Node& rhs) const {
        return cost > rhs.cost;
    }
}cur, nextnode;

void CalcFac()
{
    for (int i = 2; i < 10; i++) {
        fac[i] = fac[i - 1] * i; // 计算阶乘
    }
}

int AToInt(int a[]) // 数组状态装换成相应康拓值
{
    int x = 0;
    for (int i = 0; i < 9; i++) {
        int y = a[i];
        for (int j = 0; j < i; j++) {
        // 去掉已经用过的比 a[i] 小的值
            if (a[j] < a[i]) y--;
        }
        x += fac[8 - i] * y; // 从右往左看是第 8-i 位
    }
    return x;
}

void IntToA(int x,int *a) // 康拓值转换成数组
{
    int vvis[10];
    memset (vvis, 0, sizeof (vvis));
    for (int i = 0; i < 9; i++) {
        int y = x / fac[8 - i];
        // 第 8-i 位应该是第 y 小的数字
        for (int j = 0; j < 9; j++) {
            if (!vvis[j]) { // 没被用过的数字
                if (y == 0) { // 这时 j 就是没被用过的第 y 大的数字
                    vvis[j] = 1, a[i] = j;
                    break;
                }
                y--;
            }
        }
        x %= fac[8 - i];
    }
}

void change(int i,int flag)
{// 需要在每次 i 之后将 temp 数组还原
// 如果是每次 i 都重新调用 IntToA 来计算 temp 会 TLE
    if (flag == 1) {
        if (i == 0){
            npos = (pos + 3) % 9; // 向下移动
            swap(temp[pos], temp[npos]);
        } else if (i == 1) {
            npos = (pos + 6) % 9; // 向上移动
            swap(temp[pos], temp[npos]);
        } else if (i == 2) {
            npos = (pos + 8) % 9; // 向左移动
            swap(temp[pos], temp[npos]);
        } else if (i == 3) { // 向右移动
            npos = (pos + 1) % 9;
            swap(temp[pos], temp[npos]);
        }
    } else swap(temp[pos], temp[npos]); // 还原temp
}

int bfs()
{
    int x;
    priority_queue<Node> que;
    memset(vis, 0, sizeof (vis));
    cur.cost = 0;
    cur.cantor = st;
    que.push(cur);
    vis[st] = 1;
    int flag = 0;
    while (!que.empty()) {
        cur = que.top(); que.pop();
        if (cur.cantor == ed) {
            x = cur.cost, flag=1;
            break;
        }
        pos = cur.pos;
        IntToA(cur.cantor, temp);
        for (int i = 0; i < 4; i++) {
            change(i, 1);
            nextnode.cost = cur.cost + ((i < 2) ? cv : ch);
            nextnode.pos = npos;
            nextnode.cantor = AToInt(temp);
            if (!vis[nextnode.cantor] || vis[nextnode.cantor] > nextnode.cost) {
                vis[nextnode.cantor] = nextnode.cost;
                que.push(nextnode);
            }
            change(i, 0);
        }
    }
    return x;
}

int main()
{
    CalcFac();
    while (~scanf("%d%d", &ch, &cv)) {
        if (ch == 0 && cv == 0) break;
        for (int i = 0; i < 9; i++) {
            scanf("%d", &ast[i]);
            if (ast[i] == 0) cur.pos = i;
        }
        for (int i = 0; i < 9; i++) scanf("%d", &aed[i]);
        st = AToInt(ast);
        ed = AToInt(aed);
        ans = bfs();
        printf("%d\n", ans);
    }
    return 0;
}
\end{lstlisting}
\section{FFT}

\begin{lstlisting}
const int MAX_N = (1 << 18) + 10;
const double DPI = 2.0 * acos(-1.0);

struct CP {
	double a, b;

	CP() {}
	CP(double _a, double _b): a(_a), b(_b) {}
	CP operator * (const CP& rhs) const {
		return CP(a * rhs.a - b * rhs.b, a * rhs.b + b * rhs.a);
	}
	CP operator + (const CP& rhs) const {
		return CP(a + rhs.a, b + rhs.b);
	}
	CP operator - (const CP& rhs) const {
		return CP(a - rhs.a, b - rhs.b);
	}
} A1[MAX_N], A2[MAX_N];

int N, bit;
int rev[MAX_N];

// 初始化位逆序置换
void init(int Max) { // Max=( 变换前最高次项次数 )*2 + 1
	// 二进制平摊翻转置换
	for (N = 1, bit = 0; N < Max; N <<= 1, bit++) ;
	for (int i = 1; i < N; ++i) {
		rev[i] = (rev[i >> 1] >> 1) | ((i & 1) << (bit - 1));
	}
}

void FFT(CP *A, int n, int ope) { // ope=1:DFT, ope=-1:IDFT
	for (int i = 0; i < n; ++i) {
		// 交换互为下标反转的元素
		if (i < rev[i]) swap(A[i], A[rev[i]]);
	}
	for (int m = 2; m <= n; m <<= 1) { // 控制层数
		CP wm(cos(DPI / m), ope * sin(DPI / m)); // m 次单位复根
		for (int k = 0; k < n; k += m) { // 控制起始下标
			CP w(1, 0); // 初始化螺旋因子
			for (int j = k; j < k + (m >> 1); ++j) { // 配对
				CP t = w * A[j + (m >> 1)];
				CP u = A[j];
				A[j] = u + t;
				A[j + (m >> 1)] = u - t;
				w = w * wm; // 更新螺旋因子
			} // 上面的操作叫蝴蝶操作
		}
	}
	if (ope == -1) { // IDFT
		for (int i = 0; i < n; ++i) {
			A[i].a /= n;
		}
	}
}
\end{lstlisting}


\subsection{UVA 12633 超级车}
在一个最大$R*C(R,C\leq 50000)$的棋盘上放$n\leq 50000$个超级车,每个超级车能攻击同行 / 同列 / 同主斜线上的格子(主斜线即:左上到右下的斜线)。问放完所有超级车后有多少个格子不会被攻击到? \\

一条主对角线上的点的同一形式:$C-y+x$。把所有的没被攻击到的格子都当成是主对角线上没被攻击来统计。对被攻击到的行,列,主对角线做标记,然后没被攻击到的行列做卷积。
\begin{lstlisting}
void solve() {
    init();
    memset(A1, 0, sizeof (A1));
    memset(A2, 0, sizeof (A2));
    for (int i = 0; i < R; ++i) if (!row[i]) A1[i].a += 1;
    for (int i = 0; i < C; ++i) if (!col[i]) A2[i].a += 1;
    FFT(A1, N, 1); FFT(A2, N, 1);
    for (int i = 0; i < N; ++i) A1[i] = A1[i] * A2[i];
    FFT(A1, N, -1);
    ll ans = 0;
    for (int i = 0; i < Max; ++i) {
        if (!dia[i]) {
            ans += (ll)(A1[i].a + 0.5);
        }
    }
    printf("Case %d: %lld\n", ++cases, ans);
}

int main() {
    scanf("%d", &T);
    while (T--) {
        memset(row, 0, sizeof (row));
        memset(col, 0, sizeof (col));
        memset(dia, 0, sizeof (dia));
        scanf("%d%d%d", &R, &C, &n);
        for (int i = 1; i <= n; ++i) {
            int x, y;
            scanf("%d%d", &x, &y);
            x--, y = C - y;
            row[x] = 1, col[y] = 1;
            dia[x + y] = 1;
        }
        solve();
    }
    return 0;
}
\end{lstlisting}
\subsection{SPOJ Triple Sum}
给出$n\leq 40000$个互不相同的数$a_i\in [-20000,\ 20000]$,任取三个数(每个数不能重复取)求和,问所有可能组成的和,以及其组成方案数。 \\

先把数域平移至$[0,40000]$,然后统计每个数出现的次数,作为系数多项式$A[]$。定义$f(x)=\sum{x^{a_i}},f(2x)=\sum{x^{2*a_i}}$。
$$
\begin{aligned}
A=f(x)\bigotimes f(x)\bigotimes f(x) &=6*x^{a_{1}+a_{2}+a_{3}}+3*x^{2a_{1}+a_2}+x^{3a_1} \\
B=f(2x)\bigotimes f(x)&= x^{2a_{1}+a_2}+x^{3a_1} \\
C=f(3x)&=x^{3a_1}
\end{aligned}
$$
所以:
$$
Ans=\frac{A-3*B+2*C}{6}
$$
\begin{lstlisting}
void solve() {
    init(Max * 3 + 10);
    FFT(A1, N, 1); FFT(A2, N, 1); FFT(B, N, 1);;
    for (int i = 0; i < N; ++i) {
        A1[i] = A1[i] * A2[i];
        A1[i] = A1[i] * A2[i];
        B[i] = B[i] * A2[i];
    }
    FFT(A1, N, -1); FFT(B, N, -1);
    for (int i = 0; i < Max2; ++i) {
        if (A1[i].a <= eps) continue;
        double ans = (A1[i].a - B[i].a * 3 + C[i].a * 2) / 6.0;
        if (ans > eps) {
            printf("%d : %.0lf\n", i - 60000, ans);
        }
    }
}

int main() {
    scanf("%d", &n);
    Max = 0;
    for (int i = 0; i < n; ++i) {
        int t;
        scanf("%d", &t);
        t += 20000;
        A1[t].a += 1;
        A2[t].a += 1;
        B[2 * t].a += 1;
        C[3 * t].a += 1;
        Max = max(Max, t);
    }
    solve();
    return 0;
}
\end{lstlisting}
\subsection{HDU 5730 Shell Necklace}
给一排$n\leq 10^5$颗珍珠(不考虑成环),然后用$a_i$表示可以把连续的$i$颗珍珠染色的染色方案数,求$n$颗珍珠的染色方案数,对$313$取模。 \\

用$dp[i]$表示将$i$颗珍珠染色的染色方案数,初始化$dp[0]=1$,枚举第一串连续的珍珠的染色方案数可得:
$$
dp[i]=\sum_{j=1}^{i}a[j]*dp[i-j]
$$
用分治+FFT卷积加速,先用`double`存,最后再取模,时间复杂度:$O(n\log^2 n)$。
\begin{lstlisting}
void cdq(int left, int right) {
    if (left == right) return;
    int mid = (left + right) >> 1;
    cdq(left, mid);
    int len = right - left + 1;
    init(len);
    for (int i = 0; i <= mid - left; ++i) A1[i].a = dp[i + left] % MOD, A1[i].b = 0;
    for (int i = mid - left + 1; i < N; ++i) A1[i].a = A1[i].b = 0;
    for (int i = 0; i < N; ++i) A2[i].a = data[i], A2[i].b = 0;
    FFT(A1, N, 1); FFT(A2, N, 1);
    for (int i = 0; i < N; ++i) A1[i] = A1[i] * A2[i];
    FFT(A1, N, -1);
    for (int i = mid - left + 1; i <= right - left; ++i) {
        dp[i + left] = (dp[i + left] + (ll)(A1[i].a + 0.5) % MOD) % MOD;
    }
    cdq(mid + 1, right);
}

int main() {
    int n;
    while (~scanf("%d", &n) && n) {
        memset(data, 0, sizeof (data));
        memset(dp, 0, sizeof (dp));
        for (int i = 1; i <= n; ++i) {
            scanf("%d", &data[i]);
            data[i] %= MOD;
        }
        dp[0] = 1;
        cdq(0, n);
        printf("%d\n", dp[n]);
    }
    return 0;
}
\end{lstlisting}
\section{NTT}
模数需要满足形式:$a*2^{k}+1$,并且$k$因为尽量大的数字,然后求原根为$G$,例如:$998244353$的原根为$3$,$152076289$的原根为$106$。 \\

比较好用的素数有:$2281701377=17*2^{27}+1$,平方刚好不会爆long long;$1004535809=479*2^{21}+1$加起来刚好不会爆 int,并且这个数的原根是$3$。
\begin{lstlisting}
const int MAX_N = (1 << 18) + 10;
const int MOD = 152076289;
const int NUM = 20;
const int G = 106;
const double DPI = 2.0 * acos(-1.0);

int N, bit;
int rev[MAX_N], A1[MAX_N], A2[MAX_N], wn[2][NUM], inv[MAX_N];

int Qpow(int a, int b, int mod) {
    int ret = 1; a %= mod;
    while (b) {
        if (b & 1) ret = 1ll * ret * a % mod;
        a = 1ll * a * a % mod; b >>= 1;
    }
    return ret;
}

void init(int Max) {
    for (N = 1, bit = 0; N < Max; N <<= 1, ++bit) ;
    for (int i = 1; i < N; ++i) {
        rev[i] = (rev[i >> 1] >> 1) | ((i & 1) << (bit - 1));
    }
}

void NTT(int *A, int n, int ope) {
    for (int i = 0; i < n; ++i) if (i < rev[i]) swap(A[i], A[rev[i]]);
    int id = (ope == -1) ? 1 : 0, p = 1;
    for (int m = 2; m <= n; m <<= 1, ++p) { // m 次单位根
        for (int k = 0; k < n; k += m) {
            for (int j = k, w = 1; j < k + (m >> 1); ++j) { // 折半
                int t = 1ll * w * A[j + (m >> 1)] % MOD; // 右项
                int u = A[j] % MOD; // 左项
                A[j] = u + t, A[j + (m >> 1)] = u - t;
                if (A[j] >= MOD) A[j] -= MOD;
                if (A[j + (m >> 1)] < 0) A[j + (m >> 1)] += MOD;
                w = 1ll * w * wn[id][p] % MOD;
            }
        }
    }
    if (ope == -1) {
        for (int i = 0; i < n; ++i) {
            A[i] = 1ll * A[i] * inv[n] % MOD;
        }
    }
}

void Pre() {
    fac[0] = 1;
    for (int i = 1; i < 10010 * 2; ++i) fac[i] = 1ll * fac[i - 1] * i % MOD;
    afac[10000 * 2] = Qpow(fac[10000 * 2], MOD - 2, MOD);
    for (int i = 10000 * 2; i >= 1; --i) afac[i - 1] = 1ll * afac[i] * i % MOD;
    for (int i = 1; i < MAX_N; ++i) inv[i] = Qpow(i, MOD - 2, MOD);
    for (int i = 0; i < NUM; ++i) {
        int t = 1 << i;
        wn[0][i] = Qpow(G, (MOD - 1) / t, MOD); // 预处理求值点
        wn[1][i] = Qpow(wn[0][i], MOD - 2, MOD); // 求值点逆元
    }
}
\end{lstlisting}
\subsection{HDU 5322 Hope}
给一个$n\leq 10^5$,对于$1\sim n$的一个排列,对于$A_i$,如果存在最小的$j > i$并且$A_j > A_i$那么$i$和$j$之间有一条边,那么一种排列就会形成若干连通块,把每个连通块的大小求和为$P$,那么这个排列的贡献就是$P*P$,求$1\sim n$的所有排列的贡献和,答案对$998244353$取模。\\

用$dp[i]$表示$1\sim i$所有排列的贡献和。枚举第一个连通块的大小,第一个连通块的最后一个数字必然是$i$,初始化$dp[0]=1$,可得:
$$
dp[i]=\sum_{j=1}^{i}C_{i-1}^{j-1}*(j-1)!*j^{2}*dp[i-1]=(i-1)!*\sum_{j=1}^{i}j^2*\frac{dp[i-j]}{(i-j)!}
$$
分治预处理+NTT,时间复杂度:$O(n\log^2 n)$。
\begin{lstlisting}
void cdq(int left, int right) {
    if (left == right) return;
    int mid = (left + right) >> 1;
    cdq(left, mid);
    int len = right - left + 1;
    init(len);
    for (int i = 0; i <= mid - left; ++i) {
        A1[i] = 1ll * dp[i + left] * afac[i + left] % MOD;
    }
    for (int i = mid - left + 1; i < N; ++i) A1[i] = 0;
    for (int i = 0; i < N; ++i) A2[i] = sq[i];
    NTT(A1, N, 1); NTT(A2, N, 1);
    for (int i = 0; i < N; ++i) A1[i] = 1ll * A1[i] * A2[i] % MOD;
    NTT(A1, N, -1);
    for (int i = mid - left + 1; i <= right - left; ++i) {
        dp[i + left] += 1ll * A1[i] * fac[i + left - 1] % MOD;
        dp[i + left] %= MOD;
    }
    cdq(mid + 1, right);
}

void solve() {
    fac[0] = 1;
    for (int i = 1; i <= 100000; ++i) fac[i] = 1ll * fac[i - 1] * i % MOD;
    afac[100000] = Qpow(fac[100000], MOD - 2, MOD);
    for (int i = 100000; i >= 1; --i) afac[i - 1] = 1ll * afac[i] * i % MOD;
    for (int i = 1; i <= 100000; ++i) sq[i] = 1ll * i * i % MOD;
    for (int i = 1; i < MAX_N; ++i) inv[i] = Qpow(i, MOD - 2, MOD);
    dp[0] = 1;
    cdq(0, 100000);
}

int main() {
    solve();
    int x;
    while (~scanf("%d", &x)) {
        printf("%d\n", dp[x]);
    }
    return 0;
}
\end{lstlisting}
\subsection{HDU 5552 图计数}
给一个$n\leq 10000$和$m < 2^{31}$,表示一个$n$个点的无向图和每条边可以被染成$m$种颜色,求满足下面条件的无向图的个数:
\begin{itemize}
\item 两个点最多只有一条边
\item 至少要有一个环
\item 图连通,即图上任意两点可互达。
\end{itemize}
结果对$152076289$取模。 \\

定义$g[n]$表示$n$个点的生成图计数,可以不连通,每条边有$m+1$种选择(不取这条边和$m$种染色),一共有$\frac{n*(n-1)}{2}$条边,那么:
$$
g[n]=(m+1)^{\frac{n*(n-1)}{2}}
$$
定义$h[n]$表示$n$个点的生成树计数,根据Prufer序列和一共有$n-1$条边可得:
$$
h[n]=n^{n-2}*m^{n-1}
$$
定义$f[n]$表示$n$个点的连通图计数,那么$f[n]=$图的总数-非联通图的个数,枚举$1$号点所在连通块的大小,可得:
$$
\begin{aligned}
f[n]&=g[n]-\sum_{i=1}^{n-1}C_{n-1}^{i-1}*f[i]*g[n-i]\\
&=g[n]-(n-1)!*\sum_{i=1}^{n-1}\frac{f[i]}{(i-1)!}*\frac{g[n-i]}{(n-i)!}
\end{aligned}
$$
这是一个很明显的卷积形式,分治+NTT搞下。最后$n$个点的有环图个数=$f[n]-h[n]$。\\

时间复杂度:$O(n\log^2 n)$
\begin{lstlisting}
int T, n, m, cases = 0;
int fac[10010 * 2], afac[10010 * 2], dp[10010 * 2], g[MAX_N];

void Pre() {
    fac[0] = 1;
    for (int i = 1; i < 10010 * 2; ++i) fac[i] = 1ll * fac[i - 1] * i % MOD;
    afac[10000 * 2] = Qpow(fac[10000 * 2], MOD - 2, MOD);
    for (int i = 10000 * 2; i >= 1; --i) afac[i - 1] = 1ll * afac[i] * i % MOD;
    for (int i = 1; i < MAX_N; ++i) inv[i] = Qpow(i, MOD - 2, MOD);
    for (int i = 0; i < NUM; ++i) {
        int t = 1 << i;
        wn[0][i] = Qpow(G, (MOD - 1) / t, MOD); // 预处理求值点
        wn[1][i] = Qpow(wn[0][i], MOD - 2, MOD); // 求值点逆元
    }
}

void cdq(int left, int right) {
    if (left == right) return;
    int mid = (left + right) >> 1, len = right - left + 1;
    cdq(left, mid);
    init(len);
    for (int i = 0; i <= mid - left; ++i) A1[i] = 1ll * dp[i + left] * afac[i + left - 1] % MOD;
    for (int i = mid - left + 1; i < N; ++i) A1[i] = 0;
    for (int i = 0; i < N; ++i) A2[i] = 1ll * g[i] * afac[i] % MOD;

    NTT(A1, N, 1); NTT(A2, N, 1);
    for (int i = 0; i < N; ++i) A1[i] = 1ll * A1[i] * A2[i] % MOD;
    NTT(A1, N, -1);
    for (int i = mid - left + 1; i <= right - left; ++i) {
        dp[i + left] -= 1ll * fac[i + left - 1] * A1[i] % MOD;
        if (dp[i + left] < 0) dp[i + left] += MOD;
    }
    cdq(mid + 1, right);
}

int main() {
    Pre();
    scanf("%d", &T);
    while (T--) {
        scanf("%d%d", &n, &m);
        for (int i = 0; i < n * 2; ++i) {
            dp[i] = g[i] = Qpow(m + 1, i * (i - 1) / 2, MOD);
        }
        cdq(1, n);
        int h = 1ll * Qpow(n, n - 2, MOD) * Qpow(m, n - 1, MOD) % MOD;
        int ans = (dp[n] - h + MOD) % MOD;
        printf("Case #%d: %d\n", ++cases, ans);
    }
    return 0;
}
\end{lstlisting}
\subsection{HDU 5829 卷积平移}
给$n\leq 10^5$个数,$a_i\in [0, 10^9]$,对于这$n$个数的所有组合的非空子集$s_i$($2^{n}-1$个),从中找到最大的$\min(|s_i|,k)$个数求和,对所有的$k\in [1,n]$输出对$998244353$取模的结果。 \\

先把所有的数按照从大到小排序,考虑每个数的贡献:
$$
\begin{aligned}
dp[k]&=\sum_{i=k}^{n}(C_{i-1}^{0} + C_{i-1}^{1}+..+C_{i-1}^{k-1})*2^{n-i}*a_i \\
&=dp[k-1]+\frac{1}{(k-1)!}\sum_{i=k}^{n}((i-1)!*2^{n-i}*a[i])*\frac{1}{(i-k)!} \\
&=dp[k-1]+\frac{1}{(k-1)!}\sum_{i=k}^{n}A[i]*B[i-k]\\
\end{aligned}
$$
其中:
$$
A[i]=(i-1)!*2^{n-i}*a[i],\quad B[i-k]=\frac{1}{(i-k)!}
$$
如果把$B[i]$改为$B[i]=\frac{1}{(n-i)!}$,那么可得:
$$
\begin{aligned}
dp[n+k]&=\frac{1}{(k-1)!}\sum A[i]*B[n-i] \\
&=\frac{1}{(k-1)!}\sum A[i]*B[n-(n+k-i)] \\
&=\frac{1}{(k-1)!}\sum A[i]*B[i-k] \\
\end{aligned}
$$
这样子就和初始的式子等价了。最后别忘了前缀和。这道题提交时还需要快读。\\
时间复杂度:$O(n\log n)$
\begin{lstlisting}
int data[100010], pw[100010], fac[100010], afac[100010];

inline bool cmp(int a, int b) {
    return a > b;
}

void Pre() {
    pw[0] = fac[0] = 1;
    for (int i = 1; i < 100010; ++i) {
        pw[i] = 1ll * pw[i - 1] * 2 % MOD;
        fac[i] = 1ll * fac[i - 1] * i % MOD;
    }
    afac[100000] = Qpow(fac[100000], MOD - 2, MOD);
    for (int i = 100000; i >= 1; --i) afac[i - 1] = 1ll * afac[i] * i % MOD;
    for (int i = 1; i < MAX_N; ++i) inv[i] = Qpow(i, MOD - 2, MOD);
    for (int i = 0; i < NUM; ++i) {
        int t = 1 << i;
        wn[0][i] = Qpow(G, (MOD - 1) / t, MOD); // 预处理求值点
        wn[1][i] = Qpow(wn[0][i], MOD - 2, MOD); // 求值点逆元
    }
}

struct FastIO {
    static const int S = 1000000;
    int wpos, pos, len;
    char wbuf[S];
    FastIO(): wpos(0) {}
    inline int xchar() {
        static char buf[S];
        if (pos == len) pos = 0, len = fread(buf, 1, S, stdin);
        if (pos == len) return -1;
        return buf[pos++];
    }
    inline int xint() {
        int c = xchar(), x = 0;
        while (c <= 32 && ~c) c = xchar();
        if (c == -1) return -1;
        for (; c >= '0' && c <= '9'; c = xchar()) x = x * 10 + (c - '0');
        return x;
    }
} io;

int main() {
    Pre();
    int T, n;
    T = io.xint();
    //scanf("%d", &T);
    while (T--) {
        n = io.xint();
       // n = Read();
        for (int i = 1; i <= n; ++i)  data[i] = io.xint();
        sort(data + 1, data + n + 1, cmp);
        init(2 * n + 1);
        for (int i = 0; i < N; ++i) A1[i] = A2[i] = 0;
        for (int i = 1; i <= n; ++i) {
            A1[i] = 1ll * fac[i - 1] * pw[n - i] % MOD * data[i] % MOD;
            A2[i] = afac[n - i];
        }
        NTT(A1, N, 1); NTT(A2, N, 1);
        for (int i = 0; i < N; ++i) A1[i] = 1ll * A1[i] * A2[i] % MOD;
        NTT(A1, N, -1);
        int sum = 0;
        for (int i = n + 1; i <= 2 * n; ++i) {
            sum += 1ll * A1[i] * afac[i - n - 1] % MOD;
            if (sum >= MOD) sum -= MOD;
            printf("%d ", sum);
        }
        printf("\n");
    }
    return 0;
}
\end{lstlisting}

\section{FWT}
Fast Walsh-Hadamard Transform就是用于解决一类卷积问题的方法。大概如下:
$$
C_i=\sum_{j\oplus k=i}A_j*B_k
$$
其中$\oplus$指任一二元逻辑位运算:与,或,非等。\\
注意到位运算都是有位独立性的,每次只考虑某一位。不妨假设:
$$
A=(A_0,A_1)\quad B=(B_0, B_1)
$$
那么目标就是求出:
$$
C=A\otimes B=(\sum_{j\oplus k=0}A_j*B_k,\sum_{j\oplus k=1}A_j*B_k)
$$
假如构造了一个变换$tf$ ,满足:
$$
tf(A)\times tf(B)=tf(C)
$$
且可以在较短时间内进行变换及其逆变换,那么目标就可以实现了。
\begin{itemize}
  \item XOR
\end{itemize}
当$\oplus$指异或运算的时候,此时:
$$
C=(A_0*B_0+A_1*B_1,A_0*B_1+A_1*B_0)
$$
$tf$的变换形式:
$$
\begin{aligned}
tf(A)&=(tf(A_0)+tf(A_1),\quad tf(A_0)-tf(A_1)) \\
utf(A)&=(utf(\frac{A_0+A_1}{2}),\quad utf(\frac{A_0-A_1}{2})))
\end{aligned}
$$

\begin{itemize}
  \item AND
\end{itemize}

当$\oplus$指与运算的时候,变换为:
$$
\begin{aligned}
tf(A)&=(tf(A_0)+tf(A_1),\quad tf(A_1)) \\
utf(A)&=(utf(A_0) - utf(A_1),\quad utf(A_1))
\end{aligned}
$$

\begin{itemize}
  \item OR
\end{itemize}

当$\oplus$指或运算的时候,可以类比于与运算的变换:
$$
\begin{aligned}
tf(A)&=(tf(A_0),\quad tf(A_1)+tf(A_0)) \\
utf(A)&=(utf(A_0),\quad utf(A_1)-utf(A_0))
\end{aligned}
$$

XNOR,NAND,NOR : \\
当指异或非运算、与非运算、或非运算时。我们可以将直接用异或运算、与运算、或运算的方法求出来,然后将互反的两位交换即可。

【与运算的代码】
\begin{lstlisting}
void FWT( ll X[], int l, int r, int v ) {
    if ( l == r ) return;
    int m = ( l + r ) >> 1;
    FWT( X, l, m, v ); FWT( X, m + 1, r, v );
    FOR ( i, 0, m - l ) {
        X[ l + i ] += X[ m + 1 + i ] * v;
    }
}
\end{lstlisting}

模版:
\begin{lstlisting}
void FWT(int a[],int n)
{
    for(int d=1;d<n;d<<=1)
        for(int m=d<<1,i=0;i<n;i+=m)
            for(int j=0;j<d;j++)
            {
                int x=a[i+j],y=a[i+j+d];
                a[i+j]=(x+y)%mod,a[i+j+d]=(x-y+mod)%mod;
                //xor:a[i+j]=x+y,a[i+j+d]=(x-y+mod)%mod;
                //and:a[i+j]=x+y;
                //or:a[i+j+d]=x+y;
            }
}

void UFWT(int a[],int n)
{
    for(int d=1;d<n;d<<=1)
        for(int m=d<<1,i=0;i<n;i+=m)
            for(int j=0;j<d;j++)
            {
                int x=a[i+j],y=a[i+j+d];
                a[i+j]=1LL*(x+y)*rev%mod,a[i+j+d]=(1LL*(x-y)*rev%mod+mod)%mod;
                //xor:a[i+j]=(x+y)/2,a[i+j+d]=(x-y)/2;
                //and:a[i+j]=x-y;
                //or:a[i+j+d]=y-x;
            }
}
void solve(int a[],int b[],int n)
{
    FWT(a,n);
    FWT(b,n);
    for(int i=0;i<n;i++) a[i]=1LL*a[i]*b[i]%mod;
    UFWT(a,n);
}
\end{lstlisting}
\subsection{HDU 5909  连通块异或和为$i$个数}
给出一棵$n\leq 1000$个结点的树,每个点有一个点权,求对于每个$i\in [0,m)(m < 2^{10})$输出有多少个连通块异或和为$i$? \\

对于一个无根树的子树,如果我们钦定一个点作为无根树的根,那么对于这个树的所有子树都可以钦定出唯一的一个点作为根。 \\
考虑树$dp$ ,记$dp[i][j]$为以$i$为根,权值为$j$的子树个数。那么对于一个状态$dp[i][]$我们有加入$i$的一个子树$k$之后的新的状态的值
$$
dp[i][j]=dp[i][j]+\sum_{x\oplus y}dp[i][x]\times dp[k][y]
$$
直接这么树dp的复杂度是$O(n*m^2)$的,我们意外的发现中间的这个$dp[i][x]\times dp[k][y]$就是一个xor卷积,于是可以用FWT从$O(m^2)$优化到 $O(m\log m)$,总的复杂度优化到$O(nm\log m)$。
\begin{lstlisting}
#include<bits/stdc++.h>
using namespace std;
const int N=1e3+100,mod=1e9+7,rev=(mod+1)>>1;
// rev = Qpow(2, mod - 2, mod);

int a[N],dp[N][N],ans[N];// dp[u][i] 表示 u 为根的数, xor 值得到 i 的方案数
vector <int >mp[N];
int n,m;
int tmp[N];

void FWT(int *a,int n) {
    for(int d=1; d<n; d<<=1)
        for(int m=d<<1,i=0; i<n; i+=m)
            for(int j=0; j<d; j++) {
                int x=a[i+j],y=a[i+j+d];
                a[i+j]=(x+y)%mod,a[i+j+d]=(x-y+mod)%mod;
            }
}

void UFWT(int *a,int n)  {
    for(int d=1; d<n; d<<=1)
        for(int m=d<<1,i=0; i<n; i+=m)
            for(int j=0; j<d; j++) {
                int x=a[i+j],y=a[i+j+d];
                a[i+j]=1LL*(x+y)*rev%mod,a[i+j+d]=(1LL*(x-y)*rev%mod+mod)%mod;
            }
}

void solve(int *a,int *b,int n) {
    FWT(a,n);
    FWT(b,n);
    for(int i=0; i<n; i++)   a[i]=1LL*a[i]*b[i]%mod;
    UFWT(a,n);
}

void dfs(int x,int fa) {
    dp[x][a[x]]=1; // 自己异或值的方案数
    for (int i=0; i<mp[x].size(); i++) {
        int v=mp[x][i];
        if(v==fa) continue;
        dfs(v,x);
        for (int i=0; i<m; i++)
            tmp[i]=dp[x][i];
        solve(dp[x],dp[v],m);
        for (int i=0; i<m; i++)
            dp[x][i]=(dp[x][i]+tmp[i])%mod;
    }
    for (int i=0;i<m;i++)
        ans[i]=(ans[i]+dp[x][i])%mod;
}

int main() {
    int t,u,v;
    cin>>t;
    while(t--) {
        cin >> n >> m;
        memset(dp,0,sizeof dp);
        memset(ans,0,sizeof ans);
        for (int i=1; i<=n; i++) mp[i].clear();
        for (int i=1; i<=n; i++)  scanf("%d",&a[i]);
        for (int i=1; i<n; i++) {
            scanf("%d%d",&u,&v);
            mp[u].push_back(v);
            mp[v].push_back(u);
        }
        dfs(1,0);
        for (int i=0;i<m;i++)
            printf("%d%c",ans[i],i==m-1?'\n':' ');
    }
    return 0;
}
\end{lstlisting}

%\end{document}
