%\documentclass[UTF8]{book} % 使用book文档类型格式排版
\usepackage{ctex}  %加载包,因为我们在用中文写文档,所以必须加载这个包,否则不支持中文

%加入了一些针对XeTeX的改进并且加入了 \XeTeX 命令来输入漂亮的XeTeX logo
\usepackage{xltxtra}
%启用一些LaTeX中的功能
\usepackage{xunicode}

\usepackage{multicol}  %加载包
\usepackage{amsmath} % 调用公式宏包
\usepackage{amssymb} % 数学符号生成命令
\usepackage{array} % 数组和表格制作
\usepackage{booktabs} % 绘制水平表格线
\usepackage{calc} %四则运算
\usepackage{caption} % 插图和表格标题格式设置
\usepackage{fancyhdr} % 页眉页脚设置
\usepackage{graphicx} % 调用插图宏包
\usepackage{multicol} % 多栏排版
\usepackage{titlesec} % 章节标题格式设置

%%%% 目录样式 %%%%
\usepackage{titletoc}
\titlecontents{chapter}[1pt]{\vspace{.5\baselineskip}\bfseries}
    {{\thecontentslabel}\quad}{}
    {\hspace{.5em}\titlerule*[10pt]{$\cdot$}\contentspage}
\titlecontents{section}[2em]{\vspace{.25\baselineskip}\bfseries}
    {\thecontentslabel\quad}{}
    {\hspace{.5em}\titlerule*[10pt]{$\cdot$}\contentspage}

\usepackage{color}
\usepackage{xcolor} % 颜色处理
%\usepackage{indentfirst} % 自动首行缩进
%\setlength{\parindent}{2.22em} % 设置首行缩进的距离
% 设置超链接颜色
\usepackage[colorlinks=true,linkcolor=black,urlcolor=black,citecolor=black]{hyperref} % 根据章节标题生成PDF书签

%%%% 版面 %%%%
\usepackage[top=0.5in,bottom=0.5in,left=1.25in,right=0.8in]{geometry}
% 设置行距
\linespread{1}
\usepackage{lscape}
\usepackage{listings} %插入代码,代码页需要加入[fragile]
\usepackage{xeCJK}

%\usepackage[slantfont,boldfont]{xeCJK} % 允许斜体和粗体

%%%% fontspec 宏包 %%%%
\usepackage{fontspec}
% 指定字体
%\setmonofont[Mapping={}]{Monaco}	%英文引号之类的正常显示,相当于设置英文字体
%\setsansfont{Monaco} %设置英文字体 Monaco, Consolas,  Fantasque Sans Mono
%\setmainfont{Monaco} %设置英文字体
% \setCJKmainfont{方正兰亭黑简体}  %中文字体设置
% \setCJKsansfont{华康少女字体} %设置中文字体
% \setCJKmonofont{华康少女字体} %设置中文字体

%%%%%%%%%% 图形支持宏包 %%%%%%%%%%
\usepackage{graphicx}                % 嵌入png图像
\usepackage{color,xcolor}            % 支持彩色文本、底色、文本框等
%\usepackage{subfigure}
%\usepackage{epsfig}                 % 支持eps图像
%\usepackage{picinpar}               % 图表和文字混排宏包
%\usepackage[verbose]{wrapfig}       % 图表和文字混排宏包
%\usepackage{eso-pic}                % 向文档的部分页加n副图形, 可实现水印效果
%\usepackage{eepic}                  % 扩展的绘图支持
%\usepackage{curves}                 % 绘制复杂曲线
%\usepackage{texdraw}                % 增强的绘图工具
%\usepackage{treedoc}                % 树形图绘制
%\usepackage{pictex}                 % 可以画任意的图形
%\usepackage{hyperref}

%\setCJKmainfont{Kai}   % 设置缺省中文字体
%\setCJKmonofont{Hei}   % 设置等宽字体
%\setmainfont{Optima}   % 英文衬线字体
%\setmonofont{Monaco}   % 英文等宽字体
%\setsansfont{Trebuchet MS} % 英文无衬线字体

\makeatletter
\providecommand*\input@path{}
\newcommand\addinputpath[1]{
\expandafter\def\expandafter\input@path
\expandafter{\input@path{#1}}}
\addinputpath{body/}
\makeatother

\definecolor{keywordcolor}{rgb}{0.8,0.1,0.5}
\lstset{language=C++, %用于设置语言为C++
    numbers=left, %设置行号位置
    numberstyle=\tiny, %设置行号大小
    keywordstyle=\color{keywordcolor} \bfseries,
    identifierstyle=,
    basicstyle=\ttfamily,
    commentstyle=\color{blue} \textit, %注释颜色
    stringstyle=\ttfamily,
    showstringspaces=false,
    frame=shadowbox, %边框
    %frame = single,
    tabsize=2, %设置tab空格数
    showspaces=false, %不显示空格
    escapeinside=``, %逃逸字符(1左面的键),用于显示中文
    %breaklines, %自动折行
    captionpos=b
}

%\begin{document}

\chapter{字符串}

\section{KMP}
[HDU 1867]: 求两个字符串相加之后的最短长度。(首尾可重叠消去,两个字符串的长度$\leq 10^5$)。\\
也可以哈希做。
\begin{lstlisting}
#include <bits/stdc++.h>
using namespace std;

string s1, s2;
int nxt[100010];

string solve(string S1, string S2) {
    int n = S1.length(), m = S2.length();
    nxt[0] = nxt[1] = 0;

    int j = 0;
    for (int i = 1; i < m; ++i) {
        while (j && S2[i] != S2[j]) j = nxt[j];
        if (S2[i] == S2[j]) ++j;
        nxt[i + 1] = j;
    }

    j = 0;
    for (int i = 0; i < n; ++i) {
        while (j && S1[i] != S2[j]) j = nxt[j];
        if (S1[i] == S2[j]) ++j;
    }

    string ret(S1);
    for (int i = j; i < m; ++i) ret += S2[i];
    return ret;
}

int main() {
    while (cin >> s1 >> s2) {
        string ans1 = solve(s1, s2);
        string ans2 = solve(s2, s1);
        int len1 = ans1.length(), len2 = ans2.length();
        if (len1 < len2 || (len1 == len2 && ans1 < ans2))
            cout << ans1 << endl;
        else
            cout << ans2 << endl;
    }
    return 0;
}
\end{lstlisting}

\clearpage
\section{Shift-And算法}

基于位并行的算法,如Shift-And算法的基本思想是:将模式串集合与文本串的匹配状态用位向量存储,匹配过程就是用位操作更新位向量的过程。它的优点是所需存储空间小,匹配速度快;缺点是算法性能会随模式串个数的增多而下降,只适合中小规模的模式串集合。\\

Shift-And算法维护一个字符串的集合,集合中的每个字符串既是模式串$p$的前缀,同时也是已读入文本的后缀。每读入一个新的文本字符,该算法采用位并行的方法更新该集合,该集合用一个位掩码$D=d_{m}...d_1$来表示。$D$的第$j$位被置为1,当且仅当$p_{1}…p_{j}$是$t_{1}…t_{i}$的后缀。 \\

Shift-And算法首先构造一个$m$位($m$是模式串的长度)的向量表$B[]$,用来记录字符在模式串的出现位置。如果$p_j$为字符$c$,掩码$B[c]$的第$j$位被置为1,否则为0。首先置$D=0^m$,对于每个新读入的文本字符$t_{i+1}$,用如下公式对$D$进行更新:$D[i+1]=((D[i]<<1)|0^{m+1}1)\&B[t_{i+1}]$。在匹配时,逐个扫描文本字符并更新向量$D$,当$D[i]\&10^{m-1}≠0^{m}$时,在文本位置$i$处匹配成功。 \\

Shift-And算法扩展到多模式串时,将所有模式串的位向量$D$包装到一个机器字里,用位并行技术同时对$r$个位向量进行更新,初始化字和匹配掩码分别是所有初始化字和所有匹配掩码的连接。 \\

设机器字的长度为$w$,文本串的长度为$n$,模式串的个数为$r$,最短模式串长度为$m$,那么Shift-And算法的时间复杂度为$O(n\lceil \frac{m*r}{w}\rceil)$。由于采用了位并行技术,Shift-And算法的匹配速度是很快的。但一旦模式串的长度和超出机器字的长度,算法的性能都会发生明显下降。\\
\underline{[2016 大连B]} \\

给一个$n\leq 1000$,代表数字长度,以及每位上候选数字集合,再给一个数字字符串$s(|s|\leq 5*10^{6})$,输出$s$中所有匹配的$n$位数字子串。\\
样例输入:\\
4 (一共四位) \\
3 0 9 7 (第一位有三个候选数字分别为:0 9 7) \\
2 5 7   (第二位有两个候选数字分别为:5 7) \\
2 2 5   (第三位有两个候选数字分别为:2 5) \\
2 4 5   (第四位有两个候选数字分别为:4 5) \\
09755420524 (数字字符串s) \\
样例输出:(所有匹配的四位数字子串) \\
9755 \\
7554 \\
0524 \\

\begin{lstlisting}
#include <inttypes.h>

const int MAX_N = 1005;
const int MAX_LEN = 10000005;
const int MAX_ARR_LEN = ((MAX_N >> 6) + 5);
const int MASK = 63;

int n;
ll num[10][MAX_ARR_LEN], ret_n[MAX_ARR_LEN];
ll ind_x_arr[MAX_N], ind_y_arr[MAX_N];
char str[MAX_LEN];

void init() {
	memset(num, 0, sizeof (num));
	memset(ret_n, 0, sizeof (ret_n));
	for (int i = 0; i < n; ++i) {
		ind_x_arr[i] = (i >> 6) + 1; // i 位置属于哪一段
		ind_y_arr[i] = 1ll << (i & MASK); // 二进制中对应的位置
	}
}

inline void set_one(ll *arr, int pos) {
	arr[ind_x_arr[pos]] |= ind_y_arr[pos];
}

inline bool seek_one(ll *arr, int pos) {
	return arr[ind_x_arr[pos]] & ind_y_arr[pos];
}

inline void left_move_one(ll *arr, int pos) {
	for (int i = pos; i >= 1; --i) {
		arr[i] <<= 1;
		arr[i] |= (!!(arr[i - 1] & 0x8000000000000000ll));
	}
	set_one(arr, 0);
}

inline void and_opt(ll *arr1, ll *arr2, int pos) {
	for (int i = 1; i <= pos; ++i) {
		arr1[i] &= arr2[i];
	}
}

int main() {
	scanf("%d", &n);
	init();
	for (int i = 0; i < n; ++i) {
		int x, y;
		scanf("%d", &x);
		for (int j = 0; j < x; ++j) {
			scanf("%d", &y);
			set_one(num[y], i);
		}
	}
	scanf("%s", str);
	int len = strlen(str);
	int L = n >> 6;
	if (n & MASK) L++; // 分成若干段数
	for (int i = 0; i < len; ++i) {
		left_move_one(ret_n, L);
		and_opt(ret_n, num[str[i] - '0'], L);
		if (seek_one(ret_n, n - 1)) {
			char ch = str[i + 1];
			str[i + 1] = '\0';
			printf("%s\n", str + i - n + 1);
			str[i + 1] = ch;
		}
	}
	return 0;
}
\end{lstlisting}

\clearpage
\section{字典树}
\subsection{HDU 1857}

给出一个$n*m(n,m\in [20,500])$的字母表,对每个询问串输出它在字母表中的位置。可以从某个位置一直向右,一直向右下查找或者一直向下,输出最左上的起始位置。查找不到输出$(-1,-1)$。

先把所有的询问串创建字典树,然后扫描字母表,看字母表中的串是否有在字典树中出现。
\begin{lstlisting}
const int MAX = 1000010;
const int NUM = 26;
const int dir[4][2] = {{0, 1}, {1, 1}, {1, 0}};

int n, m;
int child[MAX][NUM], bel[MAX];
char mat[510][510];
pair<int, int> ans[10010];

struct Trie {
    int tot, root;

    void init() {
        memset(child[1], 0, sizeof (child[1]));
        tot = root = 1; bel[1] = 0;
    }
    void insert(const char* str, const int id) {
        int* cur = &root;
        for (const char* p = str; *p; ++p) {
            cur = &child[*cur][*p - 'A'];
            if (*cur == 0) {
                *cur = ++tot;
                memset(child[tot], 0, sizeof (child[tot]));
                bel[tot] = 0;
            }
        }
        bel[*cur] = id;
    }
    void query(const int x, const int y, const int id) {
        int step = 0, tmpx = x, tmpy = y;
        int* cur = &root;
        while (1) {
            if (step > 20 || tmpx >= n || tmpy >= m) return;
            char ch = mat[tmpx][tmpy];
            cur = &child[*cur][ch - 'A'];
            if (*cur == 0) return;
            int pos = bel[*cur];
            if (pos != 0 && ans[pos].first == -1) ans[pos] = make_pair(x, y);
            tmpx += dir[id][0], tmpy += dir[id][1];
            step++;
        }
    }
} dic;

int main() {
    scanf("%d%d", &n, &m);
    for (int i = 0; i < n; ++i) scanf("%s", mat[i]);
    char str[20];
    getchar(); getchar();
    int num = 0;
    dic.init();
    while (gets(str) && str[0] != '-') {
        dic.insert(str, ++num);
        ans[num] = make_pair(-1, -1);
    }
    for (int i = 0; i < n; ++i) {
        for (int j = 0; j < m; ++j) {
            for (int k = 0; k < 3; ++k) {
                dic.query(i, j, k);
            }
        }
    }
    for (int i = 1; i <= num; ++i) printf("%d %d\n", ans[i].first, ans[i].second);
    return 0;
}
\end{lstlisting}

\subsection{HDU 5536}

给$n\leq 1000$个$\leq 1e9$的正整数$a_i$,从中找到三个互不相同的$a_i,a_j,a_k$使得$(a_i+a_j)\otimes a_k$最大。输出最大值。$\otimes$表示异或。

把所有$a_i$从高位开始插进字典树,贪心查找。再支持一个删除操作就可以了,因为要保证互不相同。

时间复杂度:$O(n^2 * 30)$,再乘上一个微小的常数。
\begin{lstlisting}
const int MAX = 1000010;
const int NUM = 2;
const int MAX_N = 1010;

int child[MAX][NUM], cnt[MAX];

struct Trie {
    int root, tot;

    void init() {
        root = tot = 1;
        child[1][0] = child[1][1] = 0;
        cnt[1] = 1;
    }
    void insert(const int x) {
        int* cur = &root;
        for (int i = 30; i >= 0; --i) {
            cur = &child[*cur][(x >> i) & 1];
            if (*cur == 0) {
                *cur = ++tot;
                child[tot][0] = child[tot][1] = 0;
                cnt[tot] = 0;
            }
            cnt[*cur]++;
        }
    }
    void remove(const int x) {
        int* cur = &root;
        for (int i = 30; i >= 0; --i) {
            cur = &child[*cur][(x >> i) & 1];
            cnt[*cur]--;
        }
    }
    int query(const int x) {
        int ret = 0;
        int* cur = &root;
        for (int i = 30; i >= 0; --i) {
            int now = (x >> i) & 1, store = *cur;
            if (now == 0) {
                cur = &child[*cur][1];
                if (cnt[*cur]) ret += (1 << i);
                else cur = &child[store][0];
            } else {
                cur = &child[*cur][0];
                if (cnt[*cur]) ret += (1 << i);
                else cur = &child[store][1];
            }
        }
        return ret;
    }
} dic;

int T, n;
int a[MAX_N];

int main() {
    scanf("%d", &T);
    while (T--) {
        scanf("%d", &n);
        dic.init();
        for (int i = 1; i <= n; ++i) {
            scanf("%d", &a[i]);
            dic.insert(a[i]);
        }
        int ans = 0;
        for (int i = 1; i <= n; ++i) {
            dic.remove(a[i]);
            for (int j = i + 1; j <= n; ++j) {
                dic.remove(a[j]);
                ans = max(ans, dic.query(a[i] + a[j]));
                dic.insert(a[j]);
            }
            dic.insert(a[i]);
        }
        printf("%d\n", ans);
    }
    return 0;
}
\end{lstlisting}

%\end{document}
