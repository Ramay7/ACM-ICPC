\documentclass[UTF8]{book} % 使用book文档类型格式排版
\usepackage{ctex}  %加载包,因为我们在用中文写文档,所以必须加载这个包,否则不支持中文

%加入了一些针对XeTeX的改进并且加入了 \XeTeX 命令来输入漂亮的XeTeX logo
\usepackage{xltxtra}
%启用一些LaTeX中的功能
\usepackage{xunicode}

\usepackage{multicol}  %加载包
\usepackage{amsmath} % 调用公式宏包
\usepackage{amssymb} % 数学符号生成命令
\usepackage{array} % 数组和表格制作
\usepackage{booktabs} % 绘制水平表格线
\usepackage{calc} %四则运算
\usepackage{caption} % 插图和表格标题格式设置
\usepackage{fancyhdr} % 页眉页脚设置
\usepackage{graphicx} % 调用插图宏包
\usepackage{multicol} % 多栏排版
\usepackage{titlesec} % 章节标题格式设置

%%%% 目录样式 %%%%
\usepackage{titletoc}
\titlecontents{chapter}[1pt]{\vspace{.5\baselineskip}\bfseries}
    {{\thecontentslabel}\quad}{}
    {\hspace{.5em}\titlerule*[10pt]{$\cdot$}\contentspage}
\titlecontents{section}[2em]{\vspace{.25\baselineskip}\bfseries}
    {\thecontentslabel\quad}{}
    {\hspace{.5em}\titlerule*[10pt]{$\cdot$}\contentspage}

\usepackage{color}
\usepackage{xcolor} % 颜色处理
%\usepackage{indentfirst} % 自动首行缩进
%\setlength{\parindent}{2.22em} % 设置首行缩进的距离
% 设置超链接颜色
\usepackage[colorlinks=true,linkcolor=black,urlcolor=black,citecolor=black]{hyperref} % 根据章节标题生成PDF书签

%%%% 版面 %%%%
\usepackage[top=0.5in,bottom=0.5in,left=1.25in,right=0.8in]{geometry}
% 设置行距
\linespread{1}
\usepackage{lscape}
\usepackage{listings} %插入代码,代码页需要加入[fragile]
\usepackage{xeCJK}

%\usepackage[slantfont,boldfont]{xeCJK} % 允许斜体和粗体

%%%% fontspec 宏包 %%%%
\usepackage{fontspec}
% 指定字体
%\setmonofont[Mapping={}]{Monaco}	%英文引号之类的正常显示,相当于设置英文字体
%\setsansfont{Monaco} %设置英文字体 Monaco, Consolas,  Fantasque Sans Mono
%\setmainfont{Monaco} %设置英文字体
% \setCJKmainfont{方正兰亭黑简体}  %中文字体设置
% \setCJKsansfont{华康少女字体} %设置中文字体
% \setCJKmonofont{华康少女字体} %设置中文字体

%%%%%%%%%% 图形支持宏包 %%%%%%%%%%
\usepackage{graphicx}                % 嵌入png图像
\usepackage{color,xcolor}            % 支持彩色文本、底色、文本框等
%\usepackage{subfigure}
%\usepackage{epsfig}                 % 支持eps图像
%\usepackage{picinpar}               % 图表和文字混排宏包
%\usepackage[verbose]{wrapfig}       % 图表和文字混排宏包
%\usepackage{eso-pic}                % 向文档的部分页加n副图形, 可实现水印效果
%\usepackage{eepic}                  % 扩展的绘图支持
%\usepackage{curves}                 % 绘制复杂曲线
%\usepackage{texdraw}                % 增强的绘图工具
%\usepackage{treedoc}                % 树形图绘制
%\usepackage{pictex}                 % 可以画任意的图形
%\usepackage{hyperref}

%\setCJKmainfont{Kai}   % 设置缺省中文字体
%\setCJKmonofont{Hei}   % 设置等宽字体
%\setmainfont{Optima}   % 英文衬线字体
%\setmonofont{Monaco}   % 英文等宽字体
%\setsansfont{Trebuchet MS} % 英文无衬线字体

\makeatletter
\providecommand*\input@path{}
\newcommand\addinputpath[1]{
\expandafter\def\expandafter\input@path
\expandafter{\input@path{#1}}}
\addinputpath{body/}
\makeatother

\definecolor{keywordcolor}{rgb}{0.8,0.1,0.5}
\lstset{language=C++, %用于设置语言为C++
    numbers=left, %设置行号位置
    numberstyle=\tiny, %设置行号大小
    keywordstyle=\color{keywordcolor} \bfseries,
    identifierstyle=,
    basicstyle=\ttfamily,
    commentstyle=\color{blue} \textit, %注释颜色
    stringstyle=\ttfamily,
    showstringspaces=false,
    frame=shadowbox, %边框
    %frame = single,
    tabsize=2, %设置tab空格数
    showspaces=false, %不显示空格
    escapeinside=``, %逃逸字符(1左面的键),用于显示中文
    %breaklines, %自动折行
    captionpos=b
}