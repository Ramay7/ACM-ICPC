\documentclass[UTF8]{book} % 使用book文档类型格式排版
\usepackage{ctex}  %加载包,因为我们在用中文写文档,所以必须加载这个包,否则不支持中文

%加入了一些针对XeTeX的改进并且加入了 \XeTeX 命令来输入漂亮的XeTeX logo
\usepackage{xltxtra}
%启用一些LaTeX中的功能
\usepackage{xunicode}

\usepackage{multicol}  %加载包
\usepackage{amsmath} % 调用公式宏包
\usepackage{amssymb} % 数学符号生成命令
\usepackage{array} % 数组和表格制作
\usepackage{booktabs} % 绘制水平表格线
\usepackage{calc} %四则运算
\usepackage{caption} % 插图和表格标题格式设置
\usepackage{fancyhdr} % 页眉页脚设置
\usepackage{graphicx} % 调用插图宏包
\usepackage{multicol} % 多栏排版
\usepackage{titlesec} % 章节标题格式设置

%%%% 目录样式 %%%%
\usepackage{titletoc}
\titlecontents{chapter}[1pt]{\vspace{.5\baselineskip}\bfseries}
    {{\thecontentslabel}\quad}{}
    {\hspace{.5em}\titlerule*[10pt]{$\cdot$}\contentspage}
\titlecontents{section}[2em]{\vspace{.25\baselineskip}\bfseries}
    {\thecontentslabel\quad}{}
    {\hspace{.5em}\titlerule*[10pt]{$\cdot$}\contentspage}

\usepackage{color}
\usepackage{xcolor} % 颜色处理
%\usepackage{indentfirst} % 自动首行缩进
%\setlength{\parindent}{2.22em} % 设置首行缩进的距离
% 设置超链接颜色
\usepackage[colorlinks=true,linkcolor=black,urlcolor=black,citecolor=black]{hyperref} % 根据章节标题生成PDF书签

%%%% 版面 %%%%
\usepackage[top=0.5in,bottom=0.5in,left=1.25in,right=0.8in]{geometry}
% 设置行距
\linespread{1}
\usepackage{lscape}
\usepackage{listings} %插入代码,代码页需要加入[fragile]
\usepackage{xeCJK}

%\usepackage[slantfont,boldfont]{xeCJK} % 允许斜体和粗体

%%%% fontspec 宏包 %%%%
\usepackage{fontspec}
% 指定字体
%\setmonofont[Mapping={}]{Monaco}	%英文引号之类的正常显示,相当于设置英文字体
%\setsansfont{Monaco} %设置英文字体 Monaco, Consolas,  Fantasque Sans Mono
%\setmainfont{Monaco} %设置英文字体
% \setCJKmainfont{方正兰亭黑简体}  %中文字体设置
% \setCJKsansfont{华康少女字体} %设置中文字体
% \setCJKmonofont{华康少女字体} %设置中文字体

%%%%%%%%%% 图形支持宏包 %%%%%%%%%%
\usepackage{graphicx}                % 嵌入png图像
\usepackage{color,xcolor}            % 支持彩色文本、底色、文本框等
%\usepackage{subfigure}
%\usepackage{epsfig}                 % 支持eps图像
%\usepackage{picinpar}               % 图表和文字混排宏包
%\usepackage[verbose]{wrapfig}       % 图表和文字混排宏包
%\usepackage{eso-pic}                % 向文档的部分页加n副图形, 可实现水印效果
%\usepackage{eepic}                  % 扩展的绘图支持
%\usepackage{curves}                 % 绘制复杂曲线
%\usepackage{texdraw}                % 增强的绘图工具
%\usepackage{treedoc}                % 树形图绘制
%\usepackage{pictex}                 % 可以画任意的图形
%\usepackage{hyperref}

%\setCJKmainfont{Kai}   % 设置缺省中文字体
%\setCJKmonofont{Hei}   % 设置等宽字体
%\setmainfont{Optima}   % 英文衬线字体
%\setmonofont{Monaco}   % 英文等宽字体
%\setsansfont{Trebuchet MS} % 英文无衬线字体

\makeatletter
\providecommand*\input@path{}
\newcommand\addinputpath[1]{
\expandafter\def\expandafter\input@path
\expandafter{\input@path{#1}}}
\addinputpath{body/}
\makeatother

\definecolor{keywordcolor}{rgb}{0.8,0.1,0.5}
\lstset{language=C++, %用于设置语言为C++
    numbers=left, %设置行号位置
    numberstyle=\tiny, %设置行号大小
    keywordstyle=\color{keywordcolor} \bfseries,
    identifierstyle=,
    basicstyle=\ttfamily,
    commentstyle=\color{blue} \textit, %注释颜色
    stringstyle=\ttfamily,
    showstringspaces=false,
    frame=shadowbox, %边框
    %frame = single,
    tabsize=2, %设置tab空格数
    showspaces=false, %不显示空格
    escapeinside=``, %逃逸字符(1左面的键),用于显示中文
    %breaklines, %自动折行
    captionpos=b
}

\begin{document}

\section{苏州大学ACM集训队2016级招新笔试}

\noindent 1.一个狭长的停车场,它的宽度只能容纳一辆车,并且它只有一个口与外界相通,既是出口也是入口,这个口一次同样只能容纳一辆车进入或驶出,也就是说,最后一辆进入停车场的车会将在它之前进入的车堵在停车场内。有一天,无聊的小谈同学观察到了8辆不同的车进入且驶出了停车场,假设按照它们进入停车场的顺序给车辆编号为1、2、3、4、5、6、7、8,那么下列哪一个不可能是车辆驶出停车场的编号:$\underline{\qquad \qquad }$ \\
A. 8、7、6、5、4、3、2、1\qquad  B. 3、2、7、6、4、5、1、8 \\
C. 5、4、6、3、2、7、8、1\qquad  D. 1、2、3、4、5、6、7、8 \\

\noindent 2.以下这些小数是某些无限小数的近似结果,它们原本的数可以用很简单的方法表达,请调用你强大的想象能力把他们原本的数写出来,例如: \\
$1.4142135624=\sqrt {2}$  \\
$0.7853981634=\frac{\pi}{4}$ \\
请回答: \\
$2.7182818285=\underline{\qquad \qquad }$\\
$1.6180339887=\underline{\qquad \qquad }$\\
$0.6931471806=\underline{\qquad \qquad }$\\

\noindent 3.欧几里得算法,又称辗转相除法,最早在《几何原本》中出现,它可以用来计算两个数的最大公约数。辗转相除法根据如下定理而来,已知$a,b,c$为正整数,若$a$ 除以$b$ 余$c$,则$(a,b)=(b,c)$,其中$(x,y)$ 表示$x$ 和$y$的最大公约数。 \\

辗转相除法一般有两种写法,请根据上述过程补全代码,实现辗转相除法的python代码,其中函数$def\quad gcd(x,y)$传入两个正整数$x,y$,返回值为它们的最大公约数。
\begin{itemize}
\item 递归形式
\end{itemize}
\begin{lstlisting}
def gcd(x, y):
    if y == 0:
        return x
    else:
        return gcd(_________, _________)
\end{lstlisting}
\begin{itemize}
\item 非递归形式
\end{itemize}
\begin{lstlisting}
def gcd(x, y):
    r = 0
    while(y != 0):
        r = _________
        x = _________
        y = r
    return x
\end{lstlisting}

\noindent 4.请写出下面两段代码的运行结果
\begin{lstlisting}
import math
a = []
for i in range(0, 1000):
    a.append(i)

p = 233
left = 0
right = 999
while(left <= right):
    mid = math.floor((left + right) / 2)
    if(a[mid] > p):
        right = mid - 1
    else:
        left = mid + 1
print(a[left])
\end{lstlisting}
运行结果:$\underline{\qquad \qquad }$

\begin{lstlisting}
import math
a = []
b = []
for i in range(0, 500):
    a.append(i)
    b.append(i)
b.reverse()
for i in range(0, 500):
    a.append(b[i])

left = 0
right = 999
while(left < right):
    mid = math.floor((left + right) / 2)
    midmid = math.floor((mid + right) / 2)
    if (a[mid] > a[midmid]):
        right = midmid-1
    else:
        left = mid+1
print(a[left])
\end{lstlisting}
运行结果:$\underline{\qquad \qquad }$ \\

\noindent 5.我们把满足$x$是质数且$n\%x=0$ 的数$x$,叫做$n$的质因子,同时任何一个数都可表示成$n = p_{1}^{k_{1}}*p_{2}^{k_2}*{p_3}^{k_3}\cdots {p_t}^{k_t}$,其中$p_1,p_2\cdots p_t$为$n$含有的质因子,那么36,216,1296,7776四个数的因子数总和为:$\underline{\qquad \qquad }$,$298007187660000$的因子数个数为:$\underline{\qquad \qquad }$ \\

\noindent 6、请你找出所有的$n$,使得无论这$n$个正方形每个的边长是多少,都没有办法拼成一个正方形。$n=\underline{\qquad \qquad }$\\

\noindent 7、【2015江苏14】设向量$a_{k}=(\cos\frac{k\pi}{6},\sin \frac{k\pi}{6}+\cos \frac{k\pi}{6})$ 求$\sum_{k=1}^{6666666666}a_{k-1}\cdot a_{k} = \underline{\qquad \qquad}$($\cdot$表示点乘,结果可以保留5位有效数字,也可以用根号表示)。 \\

\noindent 8.运动会前期,你准备报名100m短跑,为了在运动会取得好成绩,你准备找香港记者进行跑步比赛,但是由于香港记者跑得实在太快了,为了能超过他的速度,你与魔法师签订契约,愿意用自己的生命交换自己的跑步速度,魔法师说你可以用10s的寿命让自己的跑步速度翻倍或者用1s的寿命使自己的跑步速度增加1m/s,现在你的速度是1m/s,而香港记者的速度是100m/s,那么你最少需要支付魔法师$\underline{\qquad \qquad }$s的寿命。 \\

\noindent 9. \\
\indent At the small zoo camel ask: "Mother Mother, why do we eyelashes so long?" Camel mother said: "When the wind came, the long
    eyelashes will enable us to be able to see the direction of the storm." small camel asked: "Mother Mother, why do we camels back then, the die ugly!" camel mother, said: "This is called hump, you can help us store a lot of water and nutrients, so that we can tolerate more than a dozen in the desert day of potable water without conditions. "small camel asked:" Mother Mother, why do we so thick the soles of the feet? "camel mother said:" That will enable us to many of the body is not stuck in soft sand, to facilitate long-distance ah journey. "small camel pleased bad:" Wow, so useful that we ah!! But mother, why we are still in zoos, do not desert hiking? "\\
\indent My talents to be useful, but now no one used. A good attitude + a successful teaching + an infinite stage = success. Each potential is unlimited, the key is to find a stage for their full potential.\\
\indent Please calculate the number of occurrences of potential in the article above. \\
$\underline{\qquad \qquad }$ \\

\noindent 10.对于一个长度为$n$的数列$A\{A[1],A[2],\cdots,A[n]\}$,它的子序列是$B\{A[b_1],A[b_2]\cdots A[b_m]\}$,其中$1
\leq b_1 < b_2 < \cdots < b_m \leq n$,即子序列是从原序列中取出若干项,他们相对顺序不变所组成的序列。 \\
如序列$\{1,3,2,4\}$,它有15个子序列 \\
$\{1\},\{3\},\{2\},\{4\},\{1,3\},\{1,2\}$ \\
$\{1,4\},\{3,2\},\{3,4\},\{2,4\},\{1,3,2\}$ \\
$\{1,3,4\},\{1,2,4\},\{3,2,4\},\{1,3,2,4\}$ \\
上升序列是该序列的项的值是递增的,如$\{1,3,5\}$。一个序列有若干上升子序列,其中子序列长度最长的称为最长上升子序列。$\{1,2,1,1,1,3,4\}$的最长上升子序列是$\{1,2,3,4\}$,长度为4。 \\

现在有一个大写英文字母序列,对于序列中的每一种字母,你可以选择一个数,并将每一个该字母用那个数替换,如此可以得到一个数列,问如何替换可使得到数列的最长上升子序列最长,你只需要输出能够得到的最长长度。 \\
例: \\
输入: \\
$ABAAAC$ \\
输出: \\
3 \\
解释: \\
$A\rightarrow 1\quad	B\rightarrow 2\quad	C\rightarrow 3\quad$得到:$121113$,最长上升子序列为$\{1,2,3\}$,长度为3 \\
$A\rightarrow 1\quad	B\rightarrow 3\quad	C\rightarrow 2\quad$得到:$131112$,最长上升子序列为$\{1,3\}$,长度为2 \\
$A\rightarrow 2\quad	B\rightarrow 1\quad	C\rightarrow 3\quad$得到:$212223$,最长上升子序列为$\{1,2,3\}$,长度为3 \\
$A\rightarrow 2\quad	B\rightarrow 3\quad	C\rightarrow 1\quad$得到:$232221$,最长上升子序列为$\{2,3\}$,长度为2 \\
$A\rightarrow 3\quad	B\rightarrow 1\quad	C\rightarrow 2\quad$得到:$313332$,最长上升子序列为$\{1,3\}$,长度为2 \\
$A\rightarrow 3\quad	B\rightarrow 2\quad	C\rightarrow 1\quad$得到:$323331$,最长上升子序列为$\{2,3\}$,长度为2 \\
还有如: \\
$A\rightarrow 1\quad	B\rightarrow 1\quad	C\rightarrow 2\quad$得到:$111112$,最长上升子序列为$\{1,2\}$,长度为2 \\
$\cdots $ \\
其中最长长度为3。 \\

\noindent 问:DWEIUDHWQIGSANPWQSCA \\
$\underline{\qquad \qquad }$ \\

\noindent 11.请问300000000内有多少个数含有9种素因子?$\underline{\qquad \qquad }$ \\

\noindent 12.我的俩个茨木又吵架了,面对一只天邪鬼黄和一只天邪鬼青,五星茨木说我的伤害是他俩血量的最小公倍数,四星茨木说我的伤害是他俩血量之和。 现在给出你五星茨木的伤害$b$和四星茨木的伤害$a$,求天邪鬼黄血量$\underline{\qquad \qquad }$ 和天邪鬼青的血量$\underline{\qquad \qquad }$. \\

\noindent 13.身为非酋的我已经抽了俩个茨木了,然而这俩个茨木却经常吵架。面对一个BOSS,他们常常争论谁能最后杀死他。 \\
已知他们的伤害都可以在$1\sim 101$之间选择,假设一个BOSS的血量为$2016$,五星茨木(茨木$A$)先出手,四星茨木(茨木$B$)后出手,问那个茨木能最后杀死BOSS$\underline{\qquad \qquad }$。(两个茨木都不是笨蛋)假设BOSS血量随机,伤害可以在$1\sim m$之间选择,则五星茨木胜利的概率为$\underline{\qquad \qquad }$,四星茨木胜利的概率为$\underline{\qquad \qquad }$。 \\

\noindent 14. \\
$
.\qquad \ \ \ \ \ \ \ 1 \\
.\qquad \ \ \ \ \ 1 \ \ 1 \\
.\qquad \ \ \ \ 1 \ \ 2\ \ 1 \\
.\qquad \ \ \ 1 \ 3 \ \ 3\ \ 1 \\
.\qquad \ \ 1 \ 4\ \ 6\ \ 4 \ \ 1 \\
.\qquad \ 1\ 5\ 10\ 10\  5\ 1 \\
$
记第一个1为第0行,往下依次编号。其中三角形左右两斜边上的数字均为1,其他位置均为其两肩上的数之和。 \\
给定任意杨辉三角的行数$n$,请输出杨辉三角中第$n$行中有$\underline{\qquad \qquad }$ 个偶数。 \\

\noindent 15.已知斐波那契数列的通项公式为
$$
a_{n}=\frac{{({\frac{1+\sqrt 5}{2}})^{n} - ({\frac{1-\sqrt 5}{2}}})^{n}}{\sqrt 5}
$$
求?????????? \\

\noindent 16.设$C[i] = a[i] – a[i-1], a[0] = 0$,求?????? \\
PS: 结果请用?????和????????的形式表示出来。\\

\noindent 17.请构造任意一组6个互不相同的正整数,使得它们相加之和等于它们的最小公倍数:$\underline{\qquad \qquad }$ \\

\noindent 18.在一个7*8的棋盘上放置5个车(可以攻击所在的行和列),求使放置的车不能互相攻击的方案数:$\underline{\qquad \qquad }$ \\

\noindent 19.设函数$f(x)$的值为斐波那契数列的第$x$项。求$f(f(f(5)))= \underline{\qquad \qquad }$\\
(注:斐波那契数列:$f(0)= f(1) = 1$,当$x > 1$时, $f(x) = f(x-1) + f(x-2)$ ) \\

\noindent 20.设函数$f(x)(x \geq 0):$
$$ f(x)=\left\{
\begin{aligned}
x \ \ \ \ \ \ \ \ \ \ &  & x \leq 2 \\
f(x-3)+f(x-2)+f(x-1) &  & x > 2 
\end{aligned}
\right.
$$
设函数$g(x) = f(x) + f(x+1) + f(x+2)$\\
设函数$k(x) = g(x) + g(x+1) + g(x+2)$ \\
求 $k(100000)\ \%\ 100007 = \underline{\qquad \qquad }$ \\

\noindent 21.数的重排 \\
现在给出$N$个两两互不相同的乱序的正整数,现在需要将这些数调换顺序,使得:$a_1 > a_2 < a_3 > a_4 < a_5 \cdots$. 以此类推的形式,$a_i$表示数列的第$i$个数。问你该如何算法,要求效率尽可能高,请给出思路(必要时给出证明)。\\
\\
\\
\\
\noindent 22.开关灯 \\
如下图所示,现在有一个3*3的网格,每个格子上有一盏灯,1代表灯亮着,0代表关着,而且每个格子上有一个按钮,每按一次这个按钮,其相邻格子的灯的状态将会反转(即:0变成1,1变成0),格子相邻的条件是:当且仅当他们有公共边。问该如何按按钮,才能使得下图的灯全部熄灭?(适当说明方法)
\begin{table}[!hbp]
\begin{tabular}{|c|c|c|}
\hline
 0 &  0 &  1 \\
\hline
 0 &  1 &  0 \\
\hline
 1 &  0 &  1 \\
\hline
\end{tabular}
\end{table} 
\\
\\
\\
\\
\\
\\
Hint:为了更好地说明题意,如上图,如果我们按了最右下角格子的按钮,则图会变成:
\begin{table}[!hbp]
\begin{tabular}{|c|c|c|}
\hline
 0 &  0 &  1 \\
\hline
 0 &  1 &  1 \\
\hline
 1 &  1 &  0 \\
\hline
\end{tabular}
\end{table} 

\end{document}