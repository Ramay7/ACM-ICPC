%\documentclass[UTF8]{book} % 使用book文档类型格式排版
\usepackage{ctex}  %加载包,因为我们在用中文写文档,所以必须加载这个包,否则不支持中文

%加入了一些针对XeTeX的改进并且加入了 \XeTeX 命令来输入漂亮的XeTeX logo
\usepackage{xltxtra}
%启用一些LaTeX中的功能
\usepackage{xunicode}

\usepackage{multicol}  %加载包
\usepackage{amsmath} % 调用公式宏包
\usepackage{amssymb} % 数学符号生成命令
\usepackage{array} % 数组和表格制作
\usepackage{booktabs} % 绘制水平表格线
\usepackage{calc} %四则运算
\usepackage{caption} % 插图和表格标题格式设置
\usepackage{fancyhdr} % 页眉页脚设置
\usepackage{graphicx} % 调用插图宏包
\usepackage{multicol} % 多栏排版
\usepackage{titlesec} % 章节标题格式设置

%%%% 目录样式 %%%%
\usepackage{titletoc}
\titlecontents{chapter}[1pt]{\vspace{.5\baselineskip}\bfseries}
    {{\thecontentslabel}\quad}{}
    {\hspace{.5em}\titlerule*[10pt]{$\cdot$}\contentspage}
\titlecontents{section}[2em]{\vspace{.25\baselineskip}\bfseries}
    {\thecontentslabel\quad}{}
    {\hspace{.5em}\titlerule*[10pt]{$\cdot$}\contentspage}

\usepackage{color}
\usepackage{xcolor} % 颜色处理
%\usepackage{indentfirst} % 自动首行缩进
%\setlength{\parindent}{2.22em} % 设置首行缩进的距离
% 设置超链接颜色
\usepackage[colorlinks=true,linkcolor=black,urlcolor=black,citecolor=black]{hyperref} % 根据章节标题生成PDF书签

%%%% 版面 %%%%
\usepackage[top=0.5in,bottom=0.5in,left=1.25in,right=0.8in]{geometry}
% 设置行距
\linespread{1}
\usepackage{lscape}
\usepackage{listings} %插入代码,代码页需要加入[fragile]
\usepackage{xeCJK}

%\usepackage[slantfont,boldfont]{xeCJK} % 允许斜体和粗体

%%%% fontspec 宏包 %%%%
\usepackage{fontspec}
% 指定字体
%\setmonofont[Mapping={}]{Monaco}	%英文引号之类的正常显示,相当于设置英文字体
%\setsansfont{Monaco} %设置英文字体 Monaco, Consolas,  Fantasque Sans Mono
%\setmainfont{Monaco} %设置英文字体
% \setCJKmainfont{方正兰亭黑简体}  %中文字体设置
% \setCJKsansfont{华康少女字体} %设置中文字体
% \setCJKmonofont{华康少女字体} %设置中文字体

%%%%%%%%%% 图形支持宏包 %%%%%%%%%%
\usepackage{graphicx}                % 嵌入png图像
\usepackage{color,xcolor}            % 支持彩色文本、底色、文本框等
%\usepackage{subfigure}
%\usepackage{epsfig}                 % 支持eps图像
%\usepackage{picinpar}               % 图表和文字混排宏包
%\usepackage[verbose]{wrapfig}       % 图表和文字混排宏包
%\usepackage{eso-pic}                % 向文档的部分页加n副图形, 可实现水印效果
%\usepackage{eepic}                  % 扩展的绘图支持
%\usepackage{curves}                 % 绘制复杂曲线
%\usepackage{texdraw}                % 增强的绘图工具
%\usepackage{treedoc}                % 树形图绘制
%\usepackage{pictex}                 % 可以画任意的图形
%\usepackage{hyperref}

%\setCJKmainfont{Kai}   % 设置缺省中文字体
%\setCJKmonofont{Hei}   % 设置等宽字体
%\setmainfont{Optima}   % 英文衬线字体
%\setmonofont{Monaco}   % 英文等宽字体
%\setsansfont{Trebuchet MS} % 英文无衬线字体

\makeatletter
\providecommand*\input@path{}
\newcommand\addinputpath[1]{
\expandafter\def\expandafter\input@path
\expandafter{\input@path{#1}}}
\addinputpath{body/}
\makeatother

\definecolor{keywordcolor}{rgb}{0.8,0.1,0.5}
\lstset{language=C++, %用于设置语言为C++
    numbers=left, %设置行号位置
    numberstyle=\tiny, %设置行号大小
    keywordstyle=\color{keywordcolor} \bfseries,
    identifierstyle=,
    basicstyle=\ttfamily,
    commentstyle=\color{blue} \textit, %注释颜色
    stringstyle=\ttfamily,
    showstringspaces=false,
    frame=shadowbox, %边框
    %frame = single,
    tabsize=2, %设置tab空格数
    showspaces=false, %不显示空格
    escapeinside=``, %逃逸字符(1左面的键),用于显示中文
    %breaklines, %自动折行
    captionpos=b
}
%\begin{document}

\chapter{STL}

\section{bitset}

作用:判断每位状态0或者1 \\
定义:bitset<MAX\_N> b; //MAX\_N 是长度下标从0开始,默认状态值都为0 \\

\subsection{用unsigned long值初始化bitset对象}

将unsigned long值转化为二进制的为模式,而bitset对象中的位集作为这种模式的副本。如果bitset类型大于unsigned long的二进制位数,则其余高阶位位置为0,如果bitset类型长度小于unsigned long值的二进制位数,则只使用unsigned long值中的低阶位,超过bitset类型长度的高阶位将会被舍弃。例如:\\
bitset<32> bs(0xffff);          // bits 0 ... 15 are set to 1; 16 ... 31 are 0

\subsection{用string对象初始化bitset对象}

string对象只能为01串。如果string对象中的字符个数小于bitset类型的长度,则高阶位为0.从string对象读入位集的顺序是从右向左(反向转化):string对象最右边的字符(即下标最大的那个字符)用来初始化低阶位(即下标为0的为)。可以指定某个子串作为初始值。例如:
\begin{lstlisting}
string str("1111111000000011001101");
bitset<32> bs1(str); //str is fully used to initialize bs1
bitset<32> bs2(str, 5, 4); // 4 bits starting at str[5], 1100
bitset<32> bs3(str, str.size() - 4);     // use last 4 characters
cout << "bs1 = " << bs1 << endl;
cout << "bs2 = " << bs2 << endl;
cout << "bs3 = " << bs3 << endl;
/*
Result:
bs1 = 00000000001111111000000011001101
bs2 = 00000000000000000000000000001100
bs3 = 00000000000000000000000000001101
*/
\end{lstlisting}
\subsection{操作}
b.any()    \qquad  \qquad        b 中是否存在置为 1 的二进制位 \\
b.none()    \qquad  \qquad      b 中不存在置为 1 的二进制位吗? \\
b.count()    \qquad  \qquad     b 中置为 1 的二进制位的个数 \\
b.size()    \qquad  \qquad      b 中二进制位的个数 \\
b[pos]      \qquad  \qquad      访问 b 中在 pos 处的二进制位 \\
b.test(pos) \qquad  \qquad      b 中在 pos 处的二进制位是否为 1 ? \\
b.set()     \qquad  \qquad      把 b 中所有二进制位都置为1  \\
b.set(pos)  \qquad  \qquad      把 b 中在pos 处的二进制位置为1 \\
b.reset()   \qquad  \qquad      把 b 中所有二进制位都置为0 \\
b.reset(pos)\qquad  \qquad      把 b 中在 pos 处的二进制位置为0 \\
b.flip()    \qquad  \qquad      把 b 中所有二进制位逐位取反 \\
b.flip(pos)  \qquad  \qquad     把 b 中在 pos 处的二进制位取反 \\
b.to\_ulong()\qquad  \qquad      用 b 中同样的二进制位返回一个 unsigned long 值 \\

\subsection{测试}


注意:使用string初始化时从右向左处理,如下初始化的各个位的值将是110,而非011 string strVal("011");
\begin{lstlisting}
bitset<3> bs1(strVal);
cout << "bs1[0] is " << bs1[0] << endl;
cout << "bs1[1] is " << bs1[1] << endl;
cout << "bs1[2] is " << bs1[2] << endl;
cout << bs1 << endl;
//any() 方法如果有一位为 1 ,则返回1
cout << "bs1.any() = " << bs1.any() << endl;
//none() 方法,如果有一个为 1 none 则返回 0 ,如果全为 0 则返回 1
bitset<3> bsNone;
cout << "bsNone.none() = " << bsNone.none() << endl;
//count() 返回几个位为1
cout << "bs1.count() = " << bs1.count() << endl;
//size() 返回位数
cout << "bs1.size() = " << bs1.size() << endl;
//flip() 诸位取反
bitset<3> bsFlip = bs1.flip();
cout << "bsFlip = " << bsFlip << endl;
//to_ulong: 用 bs1 中同样的二进制位返回一个 unsigned long 值
unsigned long val = bs1.to_ulong();
cout<< val << endl;
/*
Result:
bs1[0] is 1
bs1[1] is 1
bs1[2] is 0
011
bs1.any() = 1
bsNone.none() = 1
bs1.count() = 2
bs1.size() = 3
bsFlip = 100
4
*/
\end{lstlisting}

\clearpage
\section{单调栈、单调队列}

\subsection{单调栈}
单调队列和单调栈的时间复杂度都是:$O(n)$。 \\
单调栈主要用于解决某个元素它向左向右为最大值或最小值的最大范围是什么。如果是最大值,那就要维护单调非递增栈(唯一最大就是单调递减栈),如果是最小值就要维护单调非递减栈(唯一最小就是单调递增栈)。
\begin{lstlisting}
//求数组每个数以其为区间唯一最小值的最大区间左右端点
int top = 0, cur;
for (int i = 1; i <= n; ++i) {
    while (top) {
        cur = sta[top];
        if (data[cur] <= data[i]) break;//非唯一时去掉等号
        --top;
    }
    if (top == 0) left[i] = 1;  //data[i] 的区间左端点
    else left[i] = sta[top] + 1;
    sta[++top] = i;
}
top = 0;
for (int i = n; i >= 1; --i) {
    while (top) {
        cur = sta[top];
        if (data[cur] <= data[i]) break;
        --top;
    }
    if (top == 0) right[i] = n;
    else right[i] = sta[top] - 1;
    sta[++top] = i;
}
//如果是求以其为最大值,只需要把 <= 换为 >=
\end{lstlisting}

\underline {[POJ 3494]:给出一个$n*m$的01矩阵,求出最大全1子矩阵面积。数据范围:$n,m\leq 2000$} \\


我们把每一行单独处理,把从这行向上连续延伸全为1的最大长度看成是矩形的高,那么每行其实就是求个最大矩形面积。 \\
\begin{center}
$height[i][j]$:第$i$行第$j$列元素往上最长的连续1长度
\end{center}
需要用$O(n^2)$的复杂度预处理出$height[]$,然后需要枚举每行,每行利用单调栈可以在$O(n)$复杂度得到最大矩形面积. 总的时间复杂度是:$O(n^2)$。

\begin{lstlisting}
const int MAX_N	= 2010;

int n, m, ans;
int mat[MAX_N][MAX_N];
int height[MAX_N][MAX_N], sta[MAX_N], L[MAX_N], R[MAX_N];

//height[i][j]: 第 i 行第 j 列元素往上最长的连续 1 长度
//维护单调非递减栈
void solve(int row)
{
	int top = 0, cur;
	height[row][m + 1] = 0;
	for (int j = 1; j <= m + 1; ++j) {
		while (1) {
			cur = sta[top];
			if (height[row][cur] <= height[row][j]) break;
			R[cur] = j;
			--top;
		}
		L[j] = cur;
		sta[++top] = j;
	}
	for (int j = 1; j <= m; ++j) {
		if(mat[row][j] == 0) continue;
		int len = R[j] - L[j] - 1;
		ans = max(ans, height[row][j] * len);
	//	printf("height[%d][%d] = %d len = %d\n", row, j, height[row][j], len);
	}
}

int main()
{
	while (~scanf("%d%d", &n, &m)) {
		for (int i = 1; i <= n; ++i) {
			for (int j = 1; j <= m; ++j) {
				scanf("%d", &mat[i][j]);
			}
		}
		memset (height, 0, sizeof(height));
		for (int j = 1; j <= m; ++j) {
			for (int i = 1; i <= n; ++i) {
				if (mat[i][j] == 1) {
					height[i][j] = 1;
					while (mat[++i][j] == 1) {
						height[i][j] = height[i - 1][j] + 1;
					}
					--i;
				}
			}
		}
		ans = 0;
		for (int i = 1; i <= n; ++i) { solve(i); }
		printf("%d\n", ans);
	}
	return 0;
}
\end{lstlisting}

\subsection{单调队列}
单调队列主要用于解决满足特定条件的区间问题(如:区间最大值不超过k的最大区间长度,所有区间长度为k的最大元素值,长度不超过k的最大连续子序列和,区间最值差$\in[m,k]$的最大区间长度)。往往和前缀和结合在一起。需要判别队列应是何种单调性。\\

\underline { 【$HDU\ 3530 \ Subsequences:$】给$n(n\leq 10^5)$个数,求区间最值差$\in [m,k]$的最大区间长度。}\\


维护一个单调非递增队列和一个单调非递减队列,通过当前位置元素维护其单调性,然后调整队列首元素之差$diff$,使$diff\leq k$,判断调整后的$diff$是否满足$diff\geq m$,如果满足更新$ans$。
\begin{lstlisting}
memset(dec, 0, sizeof(dec));
memset(inc, 0, sizeof(inc));
int ans = 0, diff, pre = 0; //额外注意 pre 赋初值为0
for (int i = 0; i < n; ++i) {
    while (head_inc != tail_inc && data[i] < data[inc[tail_inc - 1]]) --tail_inc;
    inc[tail_inc++] = i;

    while (head_dec != tail_dec && data[i] > data[dec[tail_dec - 1]]) --tail_dec;
    dec[tail_dec++] = i;

    while(1) {
        diff = data[dec[head_dec]] - data[inc[head_inc]];
        if (diff > k) {
            if (dec[head_dec] < inc[head_inc]) {
                pre = dec[head_dec] + 1;  //注意区间首的位置
                head_dec++;
            } else {
                pre = inc[head_inc] + 1;
                head_inc++;
            }
        } else break;
    }
    diff = data[dec[head_dec]] - data[inc[head_inc]];
    if(diff >= m) ans = max(ans, i - pre + 1);
}
\end{lstlisting}

\section{list}

\subsection{CF 350 E}
给出一个长度为偶数的只含'('和')'并且两者个数相等的字符串,初始指针位置是$p$,下标从1开始.有三种操作:
\begin{itemize}
\item R 指针位置右移,即$p++$
\item L 指针位置左移,即$p--$
\item D 删除$p$位置和相对应括号这个区间的所有括号
\end{itemize}
输出若干次操作后的字符串. \\

需要预处理出每个括号和相对应的括号的下标.
\begin{lstlisting}
const int MAX_N = 500010;

int match[MAX_N];

int main()
{
	int n, m, p;
	while(cin >> n >> m >> p) {
		string s1, s2;
		cin >> s1;
		cin >> s2;
		stack <int> s;
		for(int i = 0; i < n; i++){
			if (s1[i] == '(') {
				s.push(i);
			} else {
				int t = s.top();
				s.pop();
				match[t] = i;
				match[i] = t;
			}
		}
		list <int> lis;
		for (int i = 0; i < n; i++) { lis.push_back(i); }
		list<int> ::iterator pos = lis.begin();
		for (int i = 1; i < p; i++) pos++;
		for (int i = 0; i < m; i++) {
			if (s2[i] == 'L') pos--;
			else if (s2[i] == 'R') pos++;
			else {
				list<int> ::iterator tmp = pos, it;
				if (s1[*pos] == '('){
					while ((*pos) != match[*tmp]) {
						pos++;
					}
				} else {
					while ((*tmp) != match[*pos]) {
						tmp--;
					}
				}
				for (it = tmp; it != pos; ){
					lis.erase(it++);
				}
				lis.erase(pos++);
				if (pos == lis.end()) pos--;
			}
		}
		
		for (it = lis.begin(); it != lis.end(); it++){
			cout << s1[*it];
		}
		cout << endl;
	}
	return 0;
}
\end{lstlisting}

%\end{document}
