\documentclass[UTF8]{book} % 使用book文档类型格式排版
\usepackage{ctex}  %加载包,因为我们在用中文写文档,所以必须加载这个包,否则不支持中文

%加入了一些针对XeTeX的改进并且加入了 \XeTeX 命令来输入漂亮的XeTeX logo
\usepackage{xltxtra}
%启用一些LaTeX中的功能
\usepackage{xunicode}

\usepackage{multicol}  %加载包
\usepackage{amsmath} % 调用公式宏包
\usepackage{amssymb} % 数学符号生成命令
\usepackage{array} % 数组和表格制作
\usepackage{booktabs} % 绘制水平表格线
\usepackage{calc} %四则运算
\usepackage{caption} % 插图和表格标题格式设置
\usepackage{fancyhdr} % 页眉页脚设置
\usepackage{graphicx} % 调用插图宏包
\usepackage{multicol} % 多栏排版
\usepackage{titlesec} % 章节标题格式设置

%%%% 目录样式 %%%%
\usepackage{titletoc}
\titlecontents{chapter}[1pt]{\vspace{.5\baselineskip}\bfseries}
    {{\thecontentslabel}\quad}{}
    {\hspace{.5em}\titlerule*[10pt]{$\cdot$}\contentspage}
\titlecontents{section}[2em]{\vspace{.25\baselineskip}\bfseries}
    {\thecontentslabel\quad}{}
    {\hspace{.5em}\titlerule*[10pt]{$\cdot$}\contentspage}

\usepackage{color}
\usepackage{xcolor} % 颜色处理
%\usepackage{indentfirst} % 自动首行缩进
%\setlength{\parindent}{2.22em} % 设置首行缩进的距离
% 设置超链接颜色
\usepackage[colorlinks=true,linkcolor=black,urlcolor=black,citecolor=black]{hyperref} % 根据章节标题生成PDF书签

%%%% 版面 %%%%
\usepackage[top=0.5in,bottom=0.5in,left=1.25in,right=0.8in]{geometry}
% 设置行距
\linespread{1}
\usepackage{lscape}
\usepackage{listings} %插入代码,代码页需要加入[fragile]
\usepackage{xeCJK}

%\usepackage[slantfont,boldfont]{xeCJK} % 允许斜体和粗体

%%%% fontspec 宏包 %%%%
\usepackage{fontspec}
% 指定字体
%\setmonofont[Mapping={}]{Monaco}	%英文引号之类的正常显示,相当于设置英文字体
%\setsansfont{Monaco} %设置英文字体 Monaco, Consolas,  Fantasque Sans Mono
%\setmainfont{Monaco} %设置英文字体
% \setCJKmainfont{方正兰亭黑简体}  %中文字体设置
% \setCJKsansfont{华康少女字体} %设置中文字体
% \setCJKmonofont{华康少女字体} %设置中文字体

%%%%%%%%%% 图形支持宏包 %%%%%%%%%%
\usepackage{graphicx}                % 嵌入png图像
\usepackage{color,xcolor}            % 支持彩色文本、底色、文本框等
%\usepackage{subfigure}
%\usepackage{epsfig}                 % 支持eps图像
%\usepackage{picinpar}               % 图表和文字混排宏包
%\usepackage[verbose]{wrapfig}       % 图表和文字混排宏包
%\usepackage{eso-pic}                % 向文档的部分页加n副图形, 可实现水印效果
%\usepackage{eepic}                  % 扩展的绘图支持
%\usepackage{curves}                 % 绘制复杂曲线
%\usepackage{texdraw}                % 增强的绘图工具
%\usepackage{treedoc}                % 树形图绘制
%\usepackage{pictex}                 % 可以画任意的图形
%\usepackage{hyperref}

%\setCJKmainfont{Kai}   % 设置缺省中文字体
%\setCJKmonofont{Hei}   % 设置等宽字体
%\setmainfont{Optima}   % 英文衬线字体
%\setmonofont{Monaco}   % 英文等宽字体
%\setsansfont{Trebuchet MS} % 英文无衬线字体

\makeatletter
\providecommand*\input@path{}
\newcommand\addinputpath[1]{
\expandafter\def\expandafter\input@path
\expandafter{\input@path{#1}}}
\addinputpath{body/}
\makeatother

\definecolor{keywordcolor}{rgb}{0.8,0.1,0.5}
\lstset{language=C++, %用于设置语言为C++
    numbers=left, %设置行号位置
    numberstyle=\tiny, %设置行号大小
    keywordstyle=\color{keywordcolor} \bfseries,
    identifierstyle=,
    basicstyle=\ttfamily,
    commentstyle=\color{blue} \textit, %注释颜色
    stringstyle=\ttfamily,
    showstringspaces=false,
    frame=shadowbox, %边框
    %frame = single,
    tabsize=2, %设置tab空格数
    showspaces=false, %不显示空格
    escapeinside=``, %逃逸字符(1左面的键),用于显示中文
    %breaklines, %自动折行
    captionpos=b
}

\begin{document}

%\chapter{数据结构}
\section{可修改的最小生成树}
给$n\leq 2*10^5$个点和$m\leq 2 * 10^5$条双向边,每条边的权值为$w_i$和单位减少代价为$c_i$,边权可以减少为负,在不超过代价$S$的条件下求出最小生成树的代价和,以及选择的边的编号和修改后的边权。 \\

先把边按照边权排序,求最小生成树。当最小生成树的所有边确定了以后,那么减少哪一条边的边权也就确定了:因为边权可以为负,那么最优一定是减少$c$最小的边。\\

然后枚举不在最小生成树中的边,对于一开始求好的最小生成树,最多只会改变一条边。对于边$i$如果要把它作为最终的最小生成树中的边,那么把这条边添加进求好的最小生成树会形成一个环,它要替换的一定是环上边权最大的边,那么问题就变成了:求树上两点路径上最大边权。利用LCA和倍增解决。\\

时间复杂度:$O(m\log m + mlog n + n\log n)$
\begin{lstlisting}
typedef long long ll;
const int MAX_N = 200010;
const int MAX_M = 200010;

int n, m, S, MinC, MinId;
int fa[MAX_N], vis[MAX_M], depth[MAX_N], anc[MAX_N][20];

struct Edge {
    int u, v, w, c, id;

    bool operator < (const Edge& rhs) const {
        return w < rhs.w;
    }
} ;

Edge edge[MAX_M], dp[MAX_N][20];
vector<Edge> g[MAX_N];

bool cmp_id(Edge a, Edge b) {
    return a.id < b.id;
}

inline int find(int x) {
    return fa[x] == x ? x : fa[x] = find(fa[x]);
}

ll Kruskal() {
    MinC = (int)(1e9) + 10;
    memset(vis, 0, sizeof (vis));
    for (int i = 0; i <= n; ++i) fa[i] = i, g[i].clear();

    sort(edge + 1, edge + m + 1);
    ll ret = 0;
    for (int i = 1; i <= m; ++i) {
        int u = edge[i].u, v = edge[i].v;
        int fu = find(u), fv = find(v);
        if (fu != fv) {
            fa[fu] = fv;
            ret += edge[i].w;
            if (MinC > edge[i].c) {
                MinC = edge[i].c;
                MinId = edge[i].id;
            }

            vis[edge[i].id] = 1;
            Edge tmp = edge[i];
            g[u].push_back(tmp);
            swap(tmp.u, tmp.v);
            g[v].push_back(tmp);
        }
    }
    return ret;
}

void dfs(int u, int p, int d) {
    depth[u] = d;
    if (u != 1) dp[u][0] = edge[p];
    for (int i = 1; i < 20; ++i) {
        anc[u][i] = anc[anc[u][i - 1]][i - 1];
        dp[u][i] = max(dp[u][i - 1], dp[anc[u][i - 1]][i - 1]);
    }
    for (int i = 0; i < g[u].size(); ++i) {
        if (g[u][i].id == p) continue;
        anc[g[u][i].v][0] = u;
        dfs(g[u][i].v, g[u][i].id, d + 1);
    }
}

Edge LCA(int u, int v) {
    if (depth[u] > depth[v]) swap(u, v);
    Edge ret;
    ret.w = -1;
    for (int i = 0; i < 20; ++i) {
        if (((depth[v] - depth[u]) >> i) & 1) {
            ret = max(ret, dp[v][i]);
            v = anc[v][i];
        }
    }
    if (u == v) return ret;
    for (int i = 19; i >= 0; --i) {
        if (anc[u][i] != anc[v][i]) {
            ret = max(ret, dp[u][i]);
            ret = max(ret, dp[v][i]);
            u = anc[u][i], v = anc[v][i];
        }
    }
    ret = max(ret, dp[u][0]);
    ret = max(ret, dp[v][0]);
    return ret;
}

void solve() {
    ll sum = Kruskal();
    ll ans = sum - S / MinC;
    sort(edge + 1, edge + m + 1, cmp_id);
    anc[1][0] = 1;
    dfs(1, -1, 0);
    int a = -1, b = MinId;
    for (int i = 1; i <= m; ++i) {
        if (vis[edge[i].id]) continue;
        Edge ee = LCA(edge[i].u, edge[i].v);
        ll tmp = sum - ee.w + (edge[i].w - S / edge[i].c);
        if (tmp < ans) {
            ans = tmp;
            a = ee.id, b = i;
        }
    }
    printf("%lld\n", ans);
    for (int i = 1; i <= m; ++i) {
        if (!vis[i] || i == a || i == b) continue;
        printf("%d %d\n", i, edge[i].w);
    }
    printf("%d %d\n", b, edge[b].w - S / edge[b].c);
}

int main() {
    scanf("%d%d", &n, &m);
    for (int i = 1; i <= m; ++i) scanf("%d", &edge[i].w);
    for (int i = 1; i <= m; ++i) scanf("%d", &edge[i].c);
    for (int i = 1; i <= m; ++i) {
        scanf("%d%d", &edge[i].u, &edge[i].v);
        edge[i].id = i;
    }
    scanf("%d", &S);
    solve();
    return 0;
}
\end{lstlisting}
\section{求曼哈顿距离最小生成树}

对每个点进行45度角分割成八个区域,每个区域只会选择一个点建边。根据对称性,对每个点只要找右半边的四个点即可。
不妨设找y轴右边45度角区域,对于点$(x_0,y_0)$应该要找满足$x_1 \geq x_0$且$y_1 - x_1 \geq y_0 - x_0$的$x_1 + y_1$最小的点$(x_1,y_1)$。 \\

因为此时曼哈顿距离为:$(x_1 - x_0)+(y_1-y_0)$。先把点坐标按照$x$排序,将$y-x$离散化,借助树状数组查询和更新区间最小值。这样子最多只会建$4*n$条边,再用$Kruskal$跑最小生成树就可以$O(n\log n)$解决了。

\begin{lstlisting}
const int MAX_N = 100010;
const int inf = 0x3f3f3f3f;

int n, edge_num, cases = 0;
int store[MAX_N], fa[MAX_N];

struct Point {
    int x, y, diff, id;
    bool operator < (const Point& rhs) const {
        return x == rhs.x ? y < rhs.y : x < rhs.x;
    }
} P[MAX_N];

struct Bit {
    int value[MAX_N], id[MAX_N];

    void init() {
        memset(value, 0x3f, sizeof (value));
        memset(id, -1, sizeof (id));
    }
    int lowbit(int x) {
        return x & -x;
    }
    void update(int x, int y, int z) {
        for (int i = x; i > 0; i -= lowbit(i)) {
            if (y < value[i]) value[i] = y, id[i] = z;
        }
    }
    pair<int, int> query(int x) {
        int ret = inf, t = -1;
        for (int i = x; i <= n; i += lowbit(i)) {
            if (value[i] < ret) ret = value[i], t = id[i];
        }
        return make_pair(ret, t);
    }
} bit;

struct Edge {
    int u, v, w;
    Edge() {}
    Edge(int _u, int _v, int _w): u(_u), v(_v), w(_w) {}

    bool operator < (const Edge& rhs) const {
        return w < rhs.w;
    }
} edge[MAX_N * 10];

inline int find(int x) {
    return fa[x] == x ? x : fa[x] = find(fa[x]);
}

ll solve() {
    edge_num = 0;
    for (int i = 0; i <= n; ++i) fa[i] = i;

    for (int dir = 1; dir <= 4; ++dir) {
        if (dir == 2 || dir == 4) {
            for (int i = 1; i <= n; ++i) swap(P[i].x, P[i].y);
        } else if (dir == 3) {
            for (int i = 1; i <= n; ++i) P[i].x = -P[i].x;
        }
        sort(P + 1, P + n + 1); // sorted by x
        for (int i = 1; i <= n; ++i) {
            store[i - 1] = P[i].diff = P[i].y - P[i].x;
        }
        sort(store, store + n);
        int tot = unique(store, store + n) - store;
        bit.init();
        for (int i = n; i >= 1; --i) {
            int pos = lower_bound(store, store + tot, P[i].diff) - store + 1;
            pair<int, int> ret = bit.query(pos);
            int a = ret.first, b = ret.second;
            if (b != -1) {
                edge[edge_num++] = Edge(P[i].id, b, a - (P[i].x + P[i].y));
            }
            bit.update(pos, P[i].x + P[i].y, P[i].id);
        }
    }

    sort(edge, edge + edge_num);
    ll ans = 0;
    for (int i = 0; i < edge_num; ++i) {
        int u = edge[i].u, v = edge[i].v, w = edge[i].w;
        int fu = find(u), fv = find(v);
        if (fu != fv) {
            ans += w;
            fa[fu] = fv;
        }
    }
    return ans;
}

int main() {
    while (~scanf("%d", &n) && n) {
        for (int i = 1; i <= n; ++i) {
            scanf("%d%d", &P[i].x, &P[i].y);
            P[i].id = i;
        }
        printf("Case %d: Total Weight = %lld\n", ++cases, solve());
    }
    return 0;
}
\end{lstlisting}

\section{其他}
\begin{lstlisting}
#include <stdio.h>
#include <string.h>
#include <math.h>
#include <algorithm>
#include <iostream>
using namespace std;
typedef long long ll;
const int MAX_N = 50010;

int n, m;
int data[MAX_N], cnt[MAX_N], pos[MAX_N];

struct Query {
    int L, R, id;
    ll nume, deno;

    bool operator < (const Query& rhs) const {
        return pos[L] < pos[rhs.L] || (pos[L] == pos[rhs.L] && R < rhs.R);
    }
    void relax() {
        ll g = __gcd(nume, deno);
        nume /= g, deno /= g;
    }
} Q[MAX_N];

bool cmp_id(const Query& a, const Query& b) {
    return a.id < b.id;
}

inline void update(int value, ll& ans, int add) {
    ans += 2 * add * cnt[value] + 1;
    cnt[value] += add;
}

void solve() {
    memset(cnt, 0, sizeof (cnt));
    int d = (int)sqrt(n + 0.5);
    for (int i = 1; i <= n; ++i) pos[i] = (i - 1) / d + 1;
    sort(Q + 1, Q + m + 1);
    int L = 1, R = 0;
    ll ans = 0;
    for (int i = 1; i <= m; ++i) {
        while (L < Q[i].L) {
            update(data[L++], ans, -1);
        }
        if (L > Q[i].L) {
            while (--L >= Q[i].L) update(data[L], ans, 1);
            ++L;
        }
        if (R < Q[i].R) {
            while (++R <= Q[i].R) update(data[R], ans, 1);
            R--;
        }
        while (R > Q[i].R) {
            update(data[R--], ans, -1);
        }
        int len = Q[i].R - Q[i].L + 1;
        Q[i].nume = ans - len;
        Q[i].deno = 1ll * len * (len - 1);
    }
    sort(Q + 1, Q + m + 1, cmp_id);
    for (int i = 1; i <= m; ++i) {
        Q[i].relax();
        printf("%lld/%lld\n", Q[i].nume, Q[i].deno);
    }
}

int main() {
    scanf("%d%d", &n, &m);
    for (int i = 1; i <= n; ++i) scanf("%d", &data[i]);
    for (int i = 1; i <= m; ++i) {
        scanf("%d%d", &Q[i].L, &Q[i].R);
        Q[i].id = i;
    }
    solve();
    return 0;
}
\end{lstlisting}


\begin{lstlisting}
#include <stdio.h>
#include <string.h>
#include <math.h>
#include <algorithm>
#include <iostream>
using namespace std;
typedef long long ll;
const int MAX_N = 100010;

int n, m;
int data[MAX_N], store[MAX_N], real[MAX_N], pos[MAX_N], cnt[MAX_N];

struct Query {
	int L, R, id;
	ll ans;

	bool operator < (const Query& rhs) const {
		return pos[L] < pos[rhs.L] || (pos[L] == pos[rhs.L] && R < rhs.R);
	}
} Q[MAX_N];

inline void update(int value, ll& ans, int add) {
	ans += 3ll * add * cnt[value] * cnt[value] + 3 * cnt[value] + add;
	cnt[value] += add;
}

bool cmp_id(const Query& a, const Query& b) {
	return a.id < b.id;
}

void solve() {
	memset(cnt, 0, sizeof (cnt));
	sort(Q + 1, Q + m + 1);
	int L = 1, R = 0;
	ll ans = 0;
	for (int i = 1; i <= m; ++i) {
		while (L < Q[i].L) {
			update(real[L++], ans, -1);
		}
		if (L > Q[i].L) {
			while (--L >= Q[i].L) update(real[L], ans, 1);
			++L;
		}
		if (R < Q[i].R) {
			while (++R <= Q[i].R) update(real[R], ans, 1);
			--R;
		}
		while (R > Q[i].R) {
			update(real[R--], ans, -1);
		}
		Q[i].ans = ans;
	}
	sort(Q + 1, Q + m + 1, cmp_id);
	for (int i = 1; i <= m; ++i) {
		printf("%I64d\n", Q[i].ans);
	}
}

int main() {
	while (~scanf("%d", &n)) {
		for (int i = 1; i <= n; ++i) {
			scanf("%d", &data[i]);
			store[i - 1] = data[i];
		}
		sort(store, store + n);
		int d = (int)sqrt(n + 0.5);
		int tot = unique(store, store + n) - store;
		for (int i = 1; i <= n; ++i) {
			real[i] = lower_bound(store, store + tot, data[i]) - store + 1;
			pos[i] = (i - 1) / d + 1;
		}	
		scanf("%d", &m);
		for (int i = 1; i <= m; ++i) {
			scanf("%d%d", &Q[i].L, &Q[i].R);
			Q[i].id = i;
		}
		solve();
	}
    return 0;
}
\end{lstlisting}
\end{document}
